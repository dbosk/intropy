\mode*
\section{Theoretical background}

The background necessary to understand the rest of this paper is presented is 
phenomenography and variation theory.
We'll briefly recap on these concepts.

\subsection{Phenomenography}

\begin{frame}
  \textcquote[pp.~112--113]{NCOL}{%
    \textins{U}nderstanding does not cause acts; instead, acts express 
    understanding%
  }.
\end{frame}
This quote captures the core of phenomenography and the essence of the type of 
data we need.

\blockcquote{NCOL}{%
  We tacitly assume that what we see is exactly what is there to be seen and 
  that others see things in the same way we do. It is hard to realize that we 
  have actually learned to see the world in certain ways and that others may 
  have learned to see it differently.
}

\begin{frame}
  \begin{block}<1-3>{Phenomenography~\parencite{Phenomenography}}
    \begin{itemize}
      \item Reality is experienced.
      \item People interpret significant aspects of reality.
    \end{itemize}
    \pause
    \begin{description}
      \item[First-order perspective] describes various aspects of the world.
      \item[Second-order perspective] describes people's experiences of various 
        aspects of the world --- phenomenography.
    \end{description}
  \end{block}

  \pause

  \begin{remark}
    \textcquote[pp.~112--113]{NCOL}{%
      \textins{U}nderstanding does not cause acts; instead, acts express 
      understanding%
    }.
  \end{remark}
\end{frame}

%\begin{frame}
%  \begin{remark}[Philosophical/theoretical view]
%    \begin{itemize}
%      \item No one sees the world \enquote{as it is}.
%      \item The world is viewed from one's own \emph{perspective}, always 
%        through the lens of the self.
%      \item \emph{The learner should learn to view things in more powerful 
%        ways.}
%    \end{itemize}
%  \end{remark}
%
%  \pause
%
%  \begin{remark}
%    \begin{itemize}
%      \item Another \emph{view} of this is \enquote{mental models}.
%    \end{itemize}
%  \end{remark}
%\end{frame}

\begin{frame}
  \begin{example}[{\cite[p.~24]{NCOL}}]
    \begin{itemize}
      \item Consider a 6-year-old child.
      \item We ask \enquote{How many fingers do you have on your left hand?}
      \item The child answers \enquote{Five}.
      \item We ask \enquote{How many fingers do you have on your right hand?}
      \item The child answers \enquote{Ten}.
    \end{itemize}
  \end{example}

  \pause

  \begin{remark}
    \begin{itemize}
      \item What does \enquote{ten} mean to the child?
    \end{itemize}
  \end{remark}
\end{frame}

\begin{frame}[fragile]
  \begin{block}{Numbers and arithmetic~\parencite{Neuman1987}}
    \begin{itemize}
      \item \Textcite{Neuman1987} studied this.
      \item Numbers has three properties required for arithmetics:
        \begin{itemize}
          \item \alert<2-3>{Ordinal (1st, 2nd, 3rd, \dots)}
          \item \alert<3>{Cardinal (1, 2, 3, \dots)}
          \item \alert<3>{Part-whole relations (\(4 = 2 + 2 = 1 + 3 = 3 + 1\))}
        \end{itemize}
        \pause
      \item The child had only discerned the ordinal property.
    \end{itemize}
  \end{block}

  \pause

  \begin{remark}
    \begin{itemize}
      \item Different children have discerned different properties (aspects).
      \item By phenomenography we can discover what the aspects are.

        \pause

      \item \alert<4>{We use the literature of misconceptions as observations.}
    \end{itemize}
  \end{remark}
\end{frame}

\subsection<article>{Definition of a misconception}

In order to summarise misconceptions found in earlier research it is 
important to define what a misconception is and how the term is used in this 
article. According to \textcite{NCOL} a misconception is where the student has 
discerned some aspects but not others (the missing aspects are called the 
critical aspects).
Misconceptions are defined in the misconceptions literature as follows: 
\textcquote[p.~1]{KumarVeerasamy2016}{\textins{a} misconception is an erroneous 
belief, which is not true or valid}. This definition is also used by 
\textcite{MisconceptionsSurvey2017}, where they specify it in a programming 
context to include aspects of syntax, concepts, control flow, learned 
constructs and debugging programs. A misconception can also include errors in 
conceptual understanding of programming. As one can see, the definition of 
misconceptions is quite broad, and will be used in this article to include all 
errors, misunderstanding, difficulties and so forth.

\subsection{Variation Theory}

\begin{frame}
  \blockcquote[p.~44]{NCOL}{%
    Imagine that you live in an entirely green world!
    Everything is equally green and is always green.
    Exactly the same nuance, no difference at all.
    Do you think that you would notice the greenness of the grass in that 
    world, and would you know what \enquote{green} is at all?
    \textelp{}
    Would it be possible for you to learn to see greenness if a teacher coming 
    from another world tried to help you by pointing to all kind of green 
    things \textelp{} saying \enquote{green} each time?%
  }
\end{frame}

Learning ← discernment ← variation \textbf{XXX This is in your slides 
Daniel, but I could not find in the book where Marton draws this 
relationship between the three.}

\begin{frame}<presentation>
  \begin{block}{Variation theory~\parencite{VariationTheory}}
    \vspace{-0.5em}
    \[
      \text{learning}
      \quad\leftarrow\quad
      \text{discernment}
      \quad\leftarrow\quad
      \text{variation}
    \]
  \end{block}

  \pause

  \begin{remark}
    \begin{itemize}
      \item There must be a pattern of variation to experience.
      \item \emph{This pattern must be experienced} for learning to happen.
    \end{itemize}
  \end{remark}
\end{frame}

Step-by-step patterns in variation theory (with the example of learning 
what the colour green is):

\begin{enumerate}
    \item Contrast (vary the critical concept) - showing a picture of a 
circle that is green, and a circle that is the colour blue.
    \item Generalisation (vary the non-critical aspect) - showing several 
figures that are green, but has different shapes. Here the green colour 
is the variant, and we have variation on the shape of the figures. 
Called induction if it comes before the pattern contrast.
    \item Fusion (vary both critical and non-critical aspect) - show 
different figures that vary in colour and in shape, to fuse the two 
earlier steps.
\end{enumerate}

\begin{frame}\label<1>{vtsummary}
  \begin{figure}
    \begin{subfigure}{0.3\columnwidth}
      \centering
      \includegraphics{fig/contrast-color.tikz}
      \caption{Contrast}
    \end{subfigure}
    \hfill
    \begin{subfigure}{0.3\columnwidth}
      \centering
      \includegraphics{fig/generalization-color.tikz}
      \caption{Generalization}
    \end{subfigure}
    \hfill
    \begin{subfigure}{0.3\columnwidth}
      \centering
      \includegraphics{fig/fusion-color.tikz}
      \caption{Fusion}
    \end{subfigure}
    \caption{%
      Illustrating the patterns of variation for aspects color and shape.
    }
  \end{figure}

  \begin{onlyenv}<1>
    \begin{example}
      \begin{itemize}
        \item Critical aspect: colour.
        \item Critical feature: blue.
        \item Non-critical feature: green.
        \item Non-critical aspects: shape, order.
        \item Non-critical features: circle, square; order of pairs.
      \end{itemize}
    \end{example}
  \end{onlyenv}
\end{frame}

\begin{frame}<presentation>
  \begin{block}{Patterns of variation}
    \vspace{-0.5em}
    \[
      \text{contrast}
      \quad\rightarrow\quad
      \text{generalization}
      \quad\rightarrow\quad
      \text{fusion}
    \]
  \end{block}

  \begin{figure}
    \begin{subfigure}{0.3\columnwidth}
      \centering
      \includegraphics{fig/contrast-color.tikz}
      \caption{Contrast}
    \end{subfigure}
    \hfill
    \begin{subfigure}{0.3\columnwidth}
      \centering
      \includegraphics{fig/generalization-color.tikz}
      \caption{Generalization}
    \end{subfigure}
    \hfill
    \begin{subfigure}{0.3\columnwidth}
      \centering
      \includegraphics{fig/fusion-color.tikz}
      \caption{Fusion}
    \end{subfigure}
    \caption{%
      Illustrating the patterns of variation for aspects color and shape.
    }
  \end{figure}
\end{frame}

\begin{frame}
  \begin{figure}
    \semitransp{%
      \begin{subfigure}{0.3\columnwidth}
        \centering
        \includegraphics{fig/contrast-color.tikz}
        \caption{Contrast}
      \end{subfigure}
    }
    \hfill
    \begin{subfigure}{0.3\columnwidth}
      \centering
      \includegraphics{fig/generalization-color.tikz}
      \caption{Generalization}
    \end{subfigure}
    \hfill
    \semitransp{%
      \begin{subfigure}{0.3\columnwidth}
        \centering
        \includegraphics{fig/fusion-color.tikz}
        \caption{Fusion}
      \end{subfigure}
      \caption{%
        Illustrating the patterns of variation for aspects color and shape.
      }
    }
  \end{figure}

  \begin{remark}
    \begin{itemize}
      \item We tend to focus on induction (\ie generalization without 
        contrast).
      \item \enquote{Good theses from previous years.}
      \item \enquote{One example of linear function, another example of linear 
        function.}
    \end{itemize}
  \end{remark}
\end{frame}

One can vary the order of these steps, but it has been found to be more 
efficient to start with the contrast.

\textbf{Things that we might need to write about here, if we want to use it 
later in the analysis}

\begin{itemize}
    \item Test in another scenario than the students have been taught in. 
For example: When learning to throw a ball, teach the students in other 
places than in the testing place.
    \item Grouping: Teaching similar concepts within a subject at the same 
time. Students learn more when being taught the differences between 
similar concepts/methods within a subject, instead of learning the 
concepts separately.
    \item Changing variables to pay attention to. Students can learn better 
if they focus on one specific attribute of a concept at a time, instead 
of trying to understand the entire concept in one iteration.
\end{itemize}








