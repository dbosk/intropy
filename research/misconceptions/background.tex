\section{Theoretical background}

\subsection{Definition of a misconception}

In order to summarise misconceptions found in earlier research it is 
important to define what a misconception is and how the term is used in this 
article. According to \textcite{NCOL} a misconception is where the student 
understand some critical aspects but misunderstand others, also defined as 
\textcquote[p.~1]{KumarVeerasamy2016}{\textins{a} misconception is an 
erroneous 
belief, which is not true or valid}. This definition is also used by 
\textcite{MisconceptionsSurvey2017}, where they specify it in a programming 
context to include aspects of syntax, concepts, control flow, learned 
constructs and debugging programs. A misconception can also include errors 
in 
conceptual understanding of programming. As one can see, the definition of 
misconceptions is broad, and will be used in this article to include all 
errors, misunderstanding, difficulties and so forth. 

\subsection{Variation Theory}

Learning ← discernment ← variation \textbf{XXX This is in your slides 
Daniel, but I could not find in the book where Marton draws this 
relationship between the three.}

Step-by-step patterns in variation theory (with the example of learning 
what the colour green is):

\begin{enumerate}
    \item Contrast (vary the critical concept) - showing a picture of a 
circle that is green, and a circle that is the colour blue.
    \item Generalisation (vary the non-critical aspect) - showing several 
figures that are green, but has different shapes. Here the green colour 
is the variant, and we have variation on the shape of the figures. 
Called induction if it comes before the pattern contrast.
    \item Fusion (vary both critical and non-critical aspect) - show 
different figures that vary in colour and in shape, to fuse the two 
earlier steps.
\end{enumerate}

One can vary the order of these steps, but it has been found to be more 
efficient to start with the contrast.

\textbf{Things that we might need to write about here, if we want to use it 
later in the analysis}

\begin{itemize}
    \item Test in another scenario than the students have been taught in. 
For example: When learning to throw a ball, teach the students in other 
places than in the testing place.
    \item Grouping: Teaching similar concepts within a subject at the same 
time. Students learn more when being taught the differences between 
similar concepts/methods within a subject, instead of learning the 
concepts separately.
    \item Changing variables to pay attention to. Students can learn better 
if they focus on one specific attribute of a concept at a time, instead 
of trying to understand the entire concept in one iteration.
\end{itemize}








