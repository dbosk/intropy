% !TEX TS-program = pdflatex
% !TEX encoding = UTF-8 Unicode

% This is a simple template for a LaTeX document using the "article" class.
% See "book", "report", "letter" for other types of document.

\documentclass[11pt]{article} % use larger type; default would be 10pt

\usepackage[utf8]{inputenc} % set input encoding (not needed with XeLaTeX)

%%% Examples of Article customizations
% These packages are optional, depending whether you want the features they provide.
% See the LaTeX Companion or other references for full information.

%%% PAGE DIMENSIONS
\usepackage{geometry} % to change the page dimensions
\geometry{a4paper} % or letterpaper (US) or a5paper or....
% \geometry{margin=2in} % for example, change the margins to 2 inches all round
% \geometry{landscape} % set up the page for landscape
%   read geometry.pdf for detailed page layout information

\usepackage{graphicx} % support the \includegraphics command and options

% \usepackage[parfill]{parskip} % Activate to begin paragraphs with an empty line rather than an indent

%%% PACKAGES
\usepackage{booktabs} % for much better looking tables
\usepackage{array} % for better arrays (eg matrices) in maths
\usepackage{paralist} % very flexible & customisable lists (eg. enumerate/itemize, etc.)
\usepackage{verbatim} % adds environment for commenting out blocks of text & for better verbatim
\usepackage{subfig} % make it possible to include more than one captioned figure/table in a single float
% These packages are all incorporated in the memoir class to one degree or another...

%%% HEADERS & FOOTERS
\usepackage{fancyhdr} % This should be set AFTER setting up the page geometry
\pagestyle{fancy} % options: empty , plain , fancy
\renewcommand{\headrulewidth}{0pt} % customise the layout...
\lhead{}\chead{}\rhead{}
\lfoot{}\cfoot{\thepage}\rfoot{}

%%% SECTION TITLE APPEARANCE
\usepackage{sectsty}
\allsectionsfont{\sffamily\mdseries\upshape} % (See the fntguide.pdf for font help)
% (This matches ConTeXt defaults)

%%% ToC (table of contents) APPEARANCE
\usepackage[nottoc,notlof,notlot]{tocbibind} % Put the bibliography in the ToC
\usepackage[titles,subfigure]{tocloft} % Alter the style of the Table of Contents
\renewcommand{\cftsecfont}{\rmfamily\mdseries\upshape}
\renewcommand{\cftsecpagefont}{\rmfamily\mdseries\upshape} % No bold!

%%% END Article customizations

%%% The "real" document content comes below...

\title{Functions in CS1}
\author{Celina Soori (at this moment)}
%\date{} % Activate to display a given date or no date (if empty),
         % otherwise the current date is printed 

\begin{document}
\maketitle

\section{Conclusion}

\subsection{Articles}

Conclusion of what the articles has been focusing on.

\subsection{Reflections}

Conclusion of all my reflections when reading the articles. 

\section{Articles}

\subsection{37 Million Compilations: Investigating Novice Programming Mistakes in Large-Scale Student Data}

\subsubsection{Abstract}
Previous investigations of student errors have typically focused on samples of hundreds of students at individual institutions. This work uses a year's worth of compilation events from over 250,000 students all over the world, taken from the large Blackbox data set. We analyze the frequency, time-to-fix, and spread of errors among users, showing how these factors inter-relate, in addition to their development over the course of the year. These results can inform the design of courses, textbooks and also tools to target the most frequent (or hardest to fix) errors.

\subsubsection{Connection to functions}
The article highlighted four misconceptions when the students are using functions and/or methods. These four are:

- A method that has a non-void return type is called and its return value ignored/discarded. For example: myObject.toString()

- Incompatible types between method return and type of variable that the value is assigned to. For example: int x = myObject.toString()

- Forgetting parentheses after a method call. For example: myObject.toString

- Invoking methods with wrong arguments (e.g. wrong types). For example: list.get("abc")\\

The first two shows that students often have a problem with understanding the meaning of input, output and return-values and how to take care of these in their code. 

The third one is a syntax-misconception which could also be connected to the input misconception. Maybe if a student have not made the connection between the parentheses and the input-values, it might be easy to forget the parentheses when they are writing the code.

The last one could be the symptoms of an earlier missunderstanding of variable types. 


\subsection{Avoiding Object Misconceptions}

\subsubsection{Abstract}
This paper has identified and characterized several misconceptions observed in students learning about object concepts, and has described simple teaching measures to avoid them. Pedagogical issues have been discussed of particular importance when constructing teaching or assessment examples.

\subsubsection{Connection to functions}
The thing that was interesting with this article that made me draw a parallell to functions from classes was the part in the article discussing the importance in showing several different examples of functions in the beginning of the course. The teacher should do this according to the article to avoid misconceptions where the students only think that a function should be constructed in one way, or only be used for one kind of problem. Maybe it is not as important as when we are teaching classes and objects, but some of the misconceptions mentioned in 2.1 could be avoided if we show enough examples in the beginning. 



\end{document}
