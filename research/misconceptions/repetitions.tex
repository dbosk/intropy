\subsection{Repetitions}

In CS1 students usually learn about repetitions, which includes for- and while-loops and in some cases recursion.

Loop constructions can be hard to trace and understand for novice students, for instance when a loop starts, ends and what is repeated and not repeated in the loop \parencite{Sekiya2013,KumarVeerasamy2016,Kaczmarczyk2010}. This was something \textcite{Sleeman1984} also realised in their article when studying high school students writing and debugging loop-structures. A common misconception that the students had was that if the loop contained a print-statement, the students thought that the only thing  repeated inside the loop was the string they saw in the terminal. The difficulties students have in tracing the code linearly when entering a loop is according to \textcite{KumarVeerasamy2016} because of the lack of understanding the students have of the looping technique and the amount of cognitive skills the tracing takes.

XXX Add analysis on how we can help students trace loops and understanding how the loop-structure works.

Another difficult part of the loop technique is to understand how an if-statement inside a loop is executed. \textcite{Sekiya2013} found in their studies that the combination of the two control structures created misconceptions. For instance the students thought that the variables in the conditional part of the loop-construction was control variables or the output from the loop. The students in the studies often got confused and started to misplace the different variables that are defined when writing an if-statement in a for-loop.

XXX Add analysis on how we can teach the combination of loops and conditionals in a way which will avoid misconceptions about the different variables used. 

