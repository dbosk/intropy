% !TEX TS-program = pdflatex
% !TEX encoding = UTF-8 Unicode

% This is a simple template for a LaTeX document using the "article" class.
% See "book", "report", "letter" for other types of document.

\documentclass[twocolumn]{article}
\usepackage[utf8]{inputenc} % set input encoding (not needed with XeLaTeX)
\usepackage[english]{babel}

\usepackage[utf8]{inputenc}
\usepackage[T1]{fontenc}
\usepackage[swedish]{babel}
\usepackage{booktabs}

\usepackage[natbib,style=alphabetic,maxbibnames=99]{biblatex}
\addbibresource{slides.bib}

\usepackage[all]{foreign}
\renewcommand{\foreignfullfont}{}
\renewcommand{\foreignabbrfont}{}

\usepackage{newclude}
\usepackage{import}

\usepackage[strict]{csquotes}
\usepackage[single]{acro}

\usepackage{subcaption}

\usepackage[noend]{algpseudocode}
\usepackage{xparse}

\let\email\texttt

\usepackage[outputdir=ltxobj]{minted}
\setminted{autogobble,fontsize=\footnotesize}

\usepackage{pythontex}
\setpythontexoutputdir{.}
\setpythontexworkingdir{..}

\usepackage{amsmath}
\usepackage{amssymb}
\usepackage{mathtools}
\usepackage{amsthm}
\usepackage{thmtools}
\usepackage[unq]{unique}
\DeclareMathOperator{\powerset}{\mathcal{P}}

\usepackage[binary-units]{siunitx}

\usepackage[capitalize]{cleveref}


\title{Student misconceptions in programming through the lens of variation 
theory}
\author{Celina Soori and Daniel Bosk (at the moment)}
%\date{} % Activate to display a given date or no date (if empty),
         % otherwise the current date is printed 

\begin{document}
\maketitle
\tableofcontents

\section{Introduction}

We study misconceptions in introductory programming from the perspective of 
variation theory.
We first survey the existing literature on misconceptions in introductory 
programming.
We then analyze these misconceptions using variation theory.

According to variation theory \parencite[Ch.~2]{NCOL}, each educational 
objective can be divided into different \emph{aspects}.
Consider the educational objective \enquote{the student should be able to use 
functions}.
One aspect of this particular educational objective is local and global scope 
of variables.
Another aspect is returning values from a function.
For a student to achieve the educational objective, she must be able to discern 
the different aspects of the educational objective.
Aspects that the student hasn't yet discerned are critical aspects.
One necessary condition for learning is that the student is introduced to a 
series of patterns of variation in the dimension of each critical aspect.
Misconceptions are examples of when a student has failed to discern (at least) 
one critical aspect.
This allows us to use misconceptions to inform our designs when designing 
teaching according to variation theory.

This study answers the following questions:
\begin{enumerate}
  \item What misconceptions in introductory programming has been identifies in 
    the literature?
  \item Based on these misconceptions, what aspects (in terms of variation 
    theory) of introductory programming can we identify?
  \item Where do we need further research?
\end{enumerate}


\section{Prerequisites: definitions}

Added this to explain different words we use in the text and how we interpreted them and use them. Maybe we dont need subsections for every word, but I started this way.

Maybe it will be needed, maybe not. Uncertain at the moment.

\subsection{Variation Theory}

Here I'm thinking we should describe what variation theory is so that we can use it in the analysis later on.

\subsection{Misconception}

Do we need to describe what we mean with misconception? Do we maybe want to use another word? 

\subsection{Computational thinking}

Maybe we want to include something about computational thinking in this article? 

\subsection{Conceptual Change Theory}

An interesting method to unlearn misconceptions and instead learn the correct conception. Mentioned by Qian \& Lehman. 


\section{Surveying the literature on misconceptions}

I'm thinking we need a method-section to describe how this research has been 
conducted. Something about the literature collecting and so on. 

Indeed, you'll need to describe the search queries and the selection criteria.


\section{Important modules in CS1}

This section is divided to reflect the different concepts that are teached during CS1. Each section will describe what students often are meant to learn and understand in that module, which is then followed with a summary of what different studies have found is difficult for students in that module. 

\subsection{Functions and variables}

Functions and variables is often the first area which is taught in 
introductory programming, an area which holds many misconceptions. This 
section will be divided into several categories, which will reflect the 
different areas that students have trouble grasping.  


\subsubsection{Conceptual understanding of variables}
According to \textcite{
Kohn2017VariableEvaluation,Plass2015Variables,Doukakis2007}, 
sometimes students believe that variables can hold an entire algorithm and 
therefore see a variable as a function (or a mathematical equation). This 
will 
create problems when a student creates a variable in belief that the 
variable 
will dynamically change its value when the equation would change its value, 
or be updated
when the variable is used in the program. Another misconception that goes 
hand in hand with the assumption that a 
variable holds an equation and not a single value, is that if in the return 
statement the student returns an equation, the student believes that the 
return 
value will be that equation, not the value that the equation represents 
\parencite{Kohn2017VariableEvaluation}.

From a variation theoretic perspective, we can say several things about this:
\begin{enumerate}
  \item That variables and functions are interconnected and should be 
treated 
    simultaneously (not \enquote{one thing at a time}), to be able to 
contrast 
    them \parencite[\cf][Ch~6, pp~167--168]{NCOL}.
  \item Unlike in mathematics, every line in a piece of program (in an 
    imperative language) constitutes a new state of the program.
    We must teach this to students through a series of patterns, as 
dictated by 
    variation theory.
\end{enumerate}

A pattern that could be used to help students understand how a variable 
defined by a mathematical statement (for example (\mintinline{python}{x = a+
b})) is interpreted by Python, could be as followed:

\begin{description}
    \item [Contrast] Show the contrast between where x first is defined by 
addition of constants, then by addition of variables, then by addition 
of variables where the value is changed after the definition of x. The 
contrast is shown by printing all these variations, where the value of 
x is invariant, but the definition varies. 
    \begin{lstlisting}[language=Python]
        def example1():
            x = 1 + 2

            return x

        def example2():
            a = 1
            b = 2
            x = a + b

            return x

        def example3():
            a = 1
            b = 2
            x = a + b
            a = 2
            b = 3

            return x
            
    \end{lstlisting}

    \item [Generalisation] In the generalisation pattern we instead let the 
value of x vary, and keep the critical aspect (the definition) 
invariant. 
    \begin{lstlisting}[language=Python]

        def example1():
            a = 1 + 1
            b = 2
            x = a + b

            return x

        def example2():
            a = 2
            b = 5 + a
            x = a + b

            return x

        def example3():
            a = 3
            b = a
            x = a + b

            return x
            
    \end{lstlisting}

    \item [Fusion] Here we will let the definition and value of x vary, and 
also use the functionality of functions (in order to teach functions 
and variables simultaneously). 

    \begin{lstlisting}[language=Python]
        def example1(a, b):
            x = a + b

            return x

        def example2(a, b):
            a = 1
            b = 2
            x = a + b

            return x

        def example3():
            a = 1
            b = 2
            x = a + b
            a = 2
            b = 3

            return x + a
            
    \end{lstlisting}
    
\end{description}



\subsubsection{Defining variables and functions}

But it is not only the right side of the variable definition that students 
can 
have misconceptions about. The name of the variable has been misunderstood 
as 
having power of the value which it holds 
\parencite{MisconceptionsSurvey2017,Sleeman1984}. For example, if a student 
names one variable \emph{max} and another variable \emph{min}, the student 
might think that the variables will strictly only hold the maximum value and 
the minimum value throughout the program, even though the code does not 
carry through this rule. 

We can both explain this phenomenon and propose a teaching design using 
variation theory.
Let's start with the explanation.
When teaching the students we always use proper (\ie relevant) variable 
names 
that relate to the purpose of the variable.
If we never show the students any examples where the variable name is not 
related to its purpose (bad variable names), they cannot separate the 
variable 
naming from its purpose.
This, inevitably, leads to the teaching design:
we must show the students that the variable names are independent of their 
purpose. To achieve this we propose this pattern:

\begin{description}
    \item [Contrast] We show a standard example, then we change a variable 
name from a relevant to an irrelevant one. We show that the program 
still works.
    \item [Generalisation] We can then \emph{generalise} this by showing 
that we can rename the other variables too, and even show other 
examples where the variable names are disconnected from the purpose.
    \item [Fusion] When done, we can point out that we name variables 
properly for readability 
(ease of comprehension), by \emph{contrasting} the same example with and 
without relevant variable names.
We can follow this by \emph{generalisation}, by showing a previously unseen 
example with unrelated variable names and trying to read it.
\end{description}


Other misconceptions that students have when defining variables were found 
by \textcite{GuoMarkelZhang2020},

\begin{enumerate}
    \item When using a variable students have the misconception that the 
      \enquote{pronoun} of the variable can be used later in the programme, 
      instead of the name that was used in the first definition of the 
      variable. For example, if a list has been defined as 
      \mintinline{python}{my_list}, the list is later referenced only as 
      \mintinline{python}{list}.

      From a variation theory perspective, this misconception can be 
managed through a similar pattern as described for the misconception 
of variable names' power of variable values. First by \emph{
contrasting} via examples where the name \mintinline{python}{my_list} 
is used throughout the programme and changed mid-programme 
respectively. Then, \emph{generalising} by using different variable 
names, but keeping them throughout the programme. The pattern ends 
with a fusion of the two.
    \item Defining a variable by using another variable students use 
      \mintinline{python}{x == y} instead of \mintinline{python}{x = y}, a 
      misconception supposedly originating from how the statement is read 
out 
      loud as \enquote{x equals y}.

      Here we propose the pattern where we first \emph{contrast} the two 
different statements, by printing the output of \mintinline{python}{x 
== y} and \mintinline{python}{x = y} separately. Both \mintinline{
python}{x} and \mintinline{python}{y} can be pre-defined, to avoid 
the programme throwing an error. However, in the contrast we could 
also include the example where an error is thrown, this to \emph{
contrast} even further. The \emph{generalisation} for this pattern is 
easy, and consists of several variable-definitions where other 
variables are used in the definition. The last pattern, \emph{Fusion} 
should consist of both variable-definitions, and where the variables 
are also used in comparison-statements. 
    \item Writing definitions of variables from left-to-right 
      (\mintinline{python}{a+b = c}) instead of right-to-left 
      (\mintinline{python}{c = a+b}). The same can be seen when using 
functions 
      in the definition of variables, for example instead of writing 
      \mintinline{python}{x=str(input())} the students write 
      \mintinline{python}{str(x) = input()}.

      This misconception can be seen as a misconception of what the 
functionality of the left and right side of a variable definition is. 
To \emph{contrast} this, we need to create code which will throw 
errors, since it is not possible to do function calls on the left 
side a variable definition. This can be done with examples from both 
misconceptions mentioned, where we do it the right way and the wrong 
way. We then \emph{generalise} it by writing several examples, with 
several function calls which will not throw errors. In this pattern 
it will however not be possible to \emph{fuse} the invariant and 
variant with each other, since the latter will throw an error for the 
programme. 
\end{enumerate}


Since Python is an interpreted language, the placement of the definition of 
a function is important, which differs from a compiled language. This gives 
room for a misconception for students that have learned to code in for 
example Java, where the definition of a function can be below a call of the 
function. 

XXX Add analysis \textbf{IF} we want to include this misconception. However 
it can be seen as out of the scope of our article, since we focus on novice 
programmers. But we still have students that might have been exposed to 
Java in high-school, so it might still be interesting to include?

\subsubsection{Arguments and return values of functions}

If we move on to the relationship between variables and functions we can see 
more misconceptions that students have; for example where input and output 
arguments 
come from and go to \parencite{Ragonis2005OOP}.
The first difficulty is how students treat return-values. When a function is 
supposed to return a value some students miss the return value, expecting 
the 
function to return it by default \parencite{Kurvinen2016,KumarVeerasamy2016}
. 
. Some students might also believe that a print-statement at the end of a 
function will act as a return statement \parencite{MisconceptionsSurvey2017}
. Furthermore, it has also been found that students that know that a return-
statement is needed have difficulties 
returning the right value or variable from a function
\parencite{KumarVeerasamy2016}. 
. 

These misconceptions are all connected to each other, and what they have in 
common is the trouble to return a value, the \emph{right} value, from a 
function. To help the students understand how to write a correct return-
statement for different functions we propose this pattern:


\begin{description}
    \item[Contrast]
    \item[Generalisation]
    \item[Fusion]
\end{description}


A student might also write a function which returns a value, but that value 
is 
not stored nor being used later in the program 
\parencite{AltadmriBrown2015}.


One can draw the conclusion that this misconception is the product of that 
students believe that a function has to have a return-statement to end, and 
that the programme will throw an error if a function misses a return-
statement. We propose this pattern to avoid this misconception:

\begin{description}
    \item[Contrast]
    \item[Generalisation]
    \item[Fusion]
\end{description}

It is not only the return value that is difficult to grasp for students, the 
input arguments are also a difficult concept for some students. When 
calling a 
function studies show that students have trouble using and understanding 
which 
arguments that are meant to be used in the function call 
\parencite{AltadmriBrown2015}. 

XXX Add analysis on how one can teach function arguments

\begin{description}
    \item[Contrast]
    \item[Generalisation]
    \item[Fusion]
\end{description}


\Textcite{Fleury1991} researched the 
misconceptions students have when using parameters in functions.
She found that students had constructed their own rules for the using 
of global and local variables which are connected to the use of variables in 
functions. These are the misconceptions she found, 

\begin{enumerate}
    \item when changing a local variable in a function, the variables with 
the same name is changed for the whole program

    \item if the local variable is not an argument in the function-call, 
the program will go back to where the function was called and search 
for it there

    \item if a function references to a global variable that is not a 
function argument, it will create an error in the program

    \item if a global variable is changed in the function body, the new 
value will not be reachable for the rest of the program if not returned 
by the function
\end{enumerate}

XXX Add analysis on how one can teach the difference between local and 
global variables and how they are used in functions. 

\begin{description}
    \item[Contrast]
    \item[Generalisation]
    \item[Fusion]
\end{description}

\subsubsection{Variables in mathematics vs programming}

The difference between variables in programming and variables in 
mathematics is something that some students do not grasp. If a student in a 
variable 
definition uses, on the right side of the equal symbol, a variable that is 
not 
defined, but the variable on the left side is already defined, they think 
that 
the computer will solve this as an equation \parencite{Plass2015Variables}. 
This assumption made by the students was also discovered by 
\textcite{Kohn2017VariableEvaluation} when giving the students the 
definition 
\mintinline{python}{x = x + 1}. If you look at this definition with a 
mathematical 
perspective you will see an unsolvable equation, which is also what some of 
the 
students saw. They did not see that the \mintinline{python}{x} to the left 
is the 
variable, 
and that the \mintinline{python}{x} to the right only holds a value. This 
definition is  
easier to understand for a novice programmer, according to 
\textcite{Kohn2017VariableEvaluation}, when we instead write this 
definition as 
\mintinline{python}{x += 1}. 

XXX Add analysis on how to help students understand the difference between 
variable definitions in programming and equations in mathematics

\begin{description}
    \item[Contrast]
    \item[Generalisation]
    \item[Fusion]
\end{description}



\subsection{Classes and objects}

\subsubsection{Role in the syllabus}

\subsubsection{Difficulties that can occur}

The concept of classes and objects in object-oriented languages is difficult and basic understanding of objects is something that many CS1 students lack (Kaczmarczyk et al., 2010). This is emphasized by Ragonis2005) who dive into this subject in their article \emph{A long-term investigation of the comprehension of OOP concepts by novices}. In their studie they noticed a number of diere misconceptions that the students had when learning about classes and objects, for instance that you can create an object from a method and that you can define a method that does not access any attributes. They also found that the students had a hard time to visualize the class as a template for a type of object, instead the students had the image of the class as a collection of objects and that the methods had the power to change, add and delete objects that are class-instances. 

Similar misconceptions has been characterized by Holland et al. (1997) in their article \emph{Avoiding object misconceptions} where they highlights the misconception that an object is a variable that can only hold one value or several values of the same type, a misconception they trace back to the first class examples that the students see. This misconception is not the only one they think is a symptome from the first classes the students see. The misconception that a class is strictly a data base is also a misconception that Holland et al. believes come from that the first classes the students write often are a good substitue to a data base and therefor shapes the student misconception. The last concept that Holland et al. discuss is the concept of storing the objects in the programme. Some students believe that the attributes of an object are the objects identifier, which leads to the misconception that there can not be two objects that have the same attributes, and therefor that one attribute of every object must be unique otherwise the programme will not be able to store it. The concept that every object has its own memory space and are stored separatly is something that is hard to grasp for some students (Holland et al., 1997; Ragonis2005). 


\subsection{Repetitions}

\subsubsection{Role in the syllabus}

\subsubsection{Difficulties that can occure}

Loop construction is an algorithm that is hard to trace and understand for novice students, for instance when a loop starts, ends and what is repeated and not repeated in the loop (Sekiya \& Yamaguchi, 2013; Veerasamy et al., 2016; Kaczmarczyk et al., 2010) . This was something Sleeman et al. (1984) also realized in their article \emph{Pascal and High-School Students: A Study of Misconceptions} when studying high school students writing and debugging loop-structures. One of the common misconception that the students had was that if the loop contained a print-statement, the students thought that the only thing that was repeated inside the loop was the string they saw in the terminal. The difficulties students have in tracing the code lineary when entering a loop is because of the lack of understanding the students have of the looping technique and the amount of cognitive skills the tracing takes (Veerasamy et al., 2016). 

Another difficult part of the loop technique is to understand how an if-statement inside a loop is executed. Sekiya \& Yamaguchi (2013) found in their studies \emph{Tracing quiz set to identify novices' programming misconceptions} that the combination of the two made the way for several student misconceptions. For instance the students thought that the variables in the conditional part of the loop-construction was control variables or the output from the loop. The students in the studies often got confused and started to misplace the different variables that are defined when writing an if-statement in a for-loop.


\subsection{Data types}

\subsubsection{Role in the syllabus}

In this chapter I was thinking that we could combine lists, arrays, maybe dictionaries, strings, charachters and the comparison of different type of variables. Maybe I will found another type that can be included here. 

\subsubsection{Difficulties that can occur}

Index in lists: Programming misconceptions in an introductory level programming course exam av Einari Kurvinen, Niko Hellgren, Erkki Kaila, Mikko-Jussi Laakso, Tapio Salakoski
Doesnt understand how lists and arrays work: Veerasamy et al., 2016 Quote \emph{According to DCI, reference to arrays versus array elements, identifying off-by-one index errors, which occur when a student using less than or equal to where is less than, and declaring and manipulating arrays as important and difficult topic for students (Goldman et al., 2008). Notably, Boulay (1986) identified that students had misconceptions with array subscripts and dimensions. We also had similar results with students confused about lists and list indices. Moreover, 13\% of the students used negative numbers as indices to read or display the values from lists.}

Comparision between different types: Programming misconceptions in an introductory level programming course exam av Einari Kurvinen, Niko Hellgren, Erkki Kaila, Mikko-Jussi Laakso, Tapio Salakoski


\mode*
\subsection{Problem solving}

\subsubsection{Role in the syllabus}

\subsubsection{Difficulties that can occur}


Here I want to have some articles about how math-problems will make it harder 
for students to solve the problem. Quote from Veerasamy et al \emph{This 
study analysis also explored that novices of programming struggled in writing 
code for math-related Questions 6 and 7 (refer Table C1). Nearly 66\% of 
students did not do well in the mathematical problem-based questions though 
explained and allowed to surf the Internet to seek for more details during 
the exam hours. A neo-Piagetian theory of cognitive development stated that 
students who are at the concrete operational stage struggle to write large 
programs with partial specifications, although they can write small programs 
from well-defined specifications (Teague et al., 2012).}

Also I would want to include difficulties students have when debugging the 
code 
and trying to find errors. Students often have a problem with tracing the 
code, 
something that is discussed by \textcite[p.~20]{Sleeman1984}. On the same 
subject as above: In what order a program will be executed in, Programming 
misconceptions in an introductory level programming course exam by Einari 
Kurvinen, Niko Hellgren, Erkki Kaila, Mikko-Jussi Laakso, Tapio Salakoski

Would also maybe like to mention how a lab instruction should be to help 
students get the right knowledge from the lab. Is discussed somewhere in 
Yizhou Qian and James Lehmans article I think.


\subsection{Conditionals}

A common control structure taught in CS1 is if- and else-statements, where 
the students learn how to create easy conditionals that controls the 
progress of the programme. How a programme will understand and execute an if
-statement is something that students have misconceptions about. A severe 
misconception that \textcite{Plass2015Variables} found was that some 
students believe that an if-statement can control if the programme will 
keep on executing or shut down, depending on if the statement is true or 
false. Students believe that an if-statement that is false will terminate 
the programme, even though a quit-statement has not been introduced. 
Another misconception is that when writing an if and else statement, both 
if and else will be executed, even when the if-statement is true \parencite{
MisconceptionsSurvey2017}.

XXX Add analysis on how we can teach conditionals in a way which help the 
students to understand when and how the code below the conditionals will be 
executed.

A syntax error that is common is when students try to chain conditions in 
an if-statement, for example "if x != a or b", where the correct statement 
should be "if x != a or x != b" \parencite{GuoMarkelZhang2020}. This 
misconception is believed to originate from the way the statement is read 
out loud as "if x is not equal to a or b", which in mathematical terms is 
the right way to state it. 

XXX Add analysis on how we can help students to grasp the way if statements 
should be stated in code, apposed to how it is stated in mathematics. 

\subsection{Things that doesn't fit in anywhere else}

IDE: What IDE is best for CS1? What difficulties can occure when using different IDEs? Should we recommend one? Quote from Qian \& Lehman \emph{Although many other syntactic-level errors are reported in previous research (see Altadmri and Brown (2015), Hristova et al. (2003), and Sorva (2012)), we do not discuss them in depth here, because problems in syntactic knowledge are often easy to detect and fix. Perhaps that is why they are often noted as the most frequent mistakes novices make (Altadmri and Brown 2015; Jackson et al. 2005). A compiler or a modern integrated development environment (IDE) may be able to find them and then provide error messages or hints for correction.}




\section{Analysis}

The analyze will be performed with the help of variation theory. 

\section{Discussion}

Here we can discuss what we want to analyze and research further. We might also want to discuss how to design the education to avoid everything above.

\newpage

\begin{thebibliography}{9}

\bibitem{texbook}
Goldman, K. Gross, P. Heeren, C. Herman, G. Kaczmarczyk, L. Loui, M. C. Zilles, C. (2010). Identifying Important and Difficult Concepts in Introductory Computing Courses using a Delphi Process. \emph{SIGCSE '08: Proceedings of the 39th SIGCSE technical symposium on Computer science education} March 2008. Pages 256–260.
\url{https://dl.acm.org/doi/10.1145/1352135.1352226} 

\bibitem{texbook}
Holland, S. Griffiths, R. Woodman, M. (1997). Avoiding Object Misconceptions. \emph{ACM SIGCSE Bulletin} Volume 29, Issue 1, March 1997. Pages 131–134.
\url{https://dl.acm.org/doi/pdf/10.1145/268084.268132} 

\bibitem{texbook}
Sekiya Y. Yamaguchi, K. (2013) Tracing quiz set to identify novices' programming misconceptions. \emph{Koli Calling '13: Proceedings of the 13th Koli Calling International Conference on Computing Education Research} November 2013. Pages 87–95.
\url{https://doi.org/10.1145/2526968.2526978} 

\end{thebibliography}
\printbibliography

\end{document}
