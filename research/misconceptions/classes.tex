\subsection{Classes and objects}

The concept of classes and objects in object-oriented languages is difficult 
and a basic understanding of objects is something that many CS1 students lack 
\parencite{Kaczmarczyk2010}. This is emphasized by \textcite{Ragonis2005OOP} 
who in their study noticed a number of dire misconceptions, two which are 
presented below. 

\begin{enumerate}
    \item An instance of a class can be created within the class' method.

    \item It is possible to define a method which does not access any of the 
class' attributes
\end{enumerate}

XXX Add analysis from variation theory


\parencite{Ragonis2005OOP} also found that students have a hard time 
visualising the class as a template for a type of object. Instead the 
students have the 
image of the class as a collection of objects and that the class' methods 
have the power to change, add and delete objects that are class-instances. 
Similar 
misconceptions has been characterised by \textcite{Holland1997}, where the 
students believe that \begin{enumerate*}
    \item an object is a variable that can only hold one or several values of 
the same type and
    \item a class is strictly a data base
\end{enumerate*}. These particular misconceptions can be traced back to the 
first classes the students write, which are often good substitutes to data 
bases 
and therefor shapes these student misconceptions. 

XXX Add analysis one how we can teach classes with easy examples and still 
manage to avoid the misconception that a class is equal to a data base. Maybe 
we can 
contrast it with the already existing objects in python (Strings, lists, 
integers etc).

The last concept that \textcite{Holland1997} discuss is the concept of 
storing the objects in the programme. Some students believe that the 
attributes of an 
object are the objects identifier, which leads to the misconception that 
there can not be two objects that have the same attributes, and therefor that 
one 
attribute of every object must be unique otherwise the programme will not be 
able to store it. The concept that every object has its own memory space and 
are 
stored separately is hard to grasp for some students \parencite{
Holland1997,Ragonis2005OOP}. 

XXX Add analysis in how we can teach how objects are stored with the help of 
variation theory.

