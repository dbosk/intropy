\subsection{Classes and objects}

\subsubsection{Role in the syllabus}

\subsubsection{Difficulties that can occur}

The concept of classes and objects in object-oriented languages is difficult and basic understanding of objects is something that many CS1 students lack (Kaczmarczyk et al., 2010). This is emphasized by Ragonis2005) who dive into this subject in their article \emph{A long-term investigation of the comprehension of OOP concepts by novices}. In their studie they noticed a number of diere misconceptions that the students had when learning about classes and objects, for instance that you can create an object from a method and that you can define a method that does not access any attributes. They also found that the students had a hard time to visualize the class as a template for a type of object, instead the students had the image of the class as a collection of objects and that the methods had the power to change, add and delete objects that are class-instances. 

Similar misconceptions has been characterized by Holland et al. (1997) in their article \emph{Avoiding object misconceptions} where they highlights the misconception that an object is a variable that can only hold one value or several values of the same type, a misconception they trace back to the first class examples that the students see. This misconception is not the only one they think is a symptome from the first classes the students see. The misconception that a class is strictly a data base is also a misconception that Holland et al. believes come from that the first classes the students write often are a good substitue to a data base and therefor shapes the student misconception. The last concept that Holland et al. discuss is the concept of storing the objects in the programme. Some students believe that the attributes of an object are the objects identifier, which leads to the misconception that there can not be two objects that have the same attributes, and therefor that one attribute of every object must be unique otherwise the programme will not be able to store it. The concept that every object has its own memory space and are stored separatly is something that is hard to grasp for some students (Holland et al., 1997; Ragonis2005). 

