\section{Misconceptions we have found}

\subsection{Misconceptions not in articles}

    \begin{itemize}
        \item That in order to return a the value of a variable from a 
function that variable must be an input argument to the function.
    
        Example:
            \hfill
            \begin{minted}{python}
            def foo(x):
                x = 3
                return x
        
            x = 0
            x = foo(x)
            \end{minted}
            \hfill

        \item That the built-in function \mintinline{python}{input()} 
translates the given value. For example, if the user writes a 
number, \mintinline{python}{input()} will return an integer. 

        \item That functions will be automatically called in the correct 
order, without the functions being invoked in the programme. 

        \item That in order to invoke a function several times if a 
condition is true, the function need to call itself instead of 
having a loop-structure that repeatedly invokes the function. 

        Example:
            \hfill
            \begin{minipage}[t]{0.45\columnwidth}
            \begin{minted}{python}
                def foo():
                    try:
                        x = int(input("Integer:"))
                        return x
                    except ValueError:
                        foo()
            \end{minted}
            \end{minipage}
            \hfill
            \begin{minipage}[t]{0.45\columnwidth}
            \begin{minted}{python}
                def foo():
                    while True:
                        try:
                            x = int(input("Integer:"))
                            return x
                        except ValueError:
                            continue
            \end{minted}
            \end{minipage}
        \item That even though several arguments are stated in a function 
definition, when invoking the function the parameters does not have 
to be passed to the function if the arguments' names are global 
variables in the programme.

        Example:

        \begin{minted}{python}

            def foo(x,y):
                print(x+y)

            x = 3
            y = 4

            foo()
            
        \end{minted}

        \item That a return-statement in a function will "send" the 
variable name and value to the place the function was invoked in, 
and that the variable name then later can be used by the programme.

        Example:
        \hfill
        \begin{minted}{python}
            def foo():
                x = 3
    
                return x
    
            foo()
            print(x)
        \end{minted}
    \end{itemize}

\subsection{Misconceptions already mentioned}

    \begin{itemize}
        \item That a variable name controls what value that can be assigned 
to it. For example that a programme will throw an error if an 
integer is assigned to a variable called \mintinline{python}{my_
string}.
        \item That a local variable is reachable outside the function it is 
defined in.
        \item That a while-loop without a condition will terminate without 
a \mintinline{python}{break}-statement.
        

        
    \end{itemize}

\subsection{Difficulties that can not be translated to a misconception}

\begin{itemize}
    \item The word "iterate" need to be explained more clearly, not a word 
all students have heard before. 
\end{itemize}