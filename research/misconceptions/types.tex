\subsection{Data types}

\subsubsection{Role in the syllabus}

In this chapter I was thinking that we could combine lists, arrays, maybe dictionaries, strings, charachters and the comparison of different type of variables. Maybe I will found another type that can be included here. 

\subsubsection{Difficulties that can occur}

A data type that is common in CS1 is arrays. \textcite{Kurvinen2016} found in their studies that it was not intuitive for the students that the index of an array starts at 0 instead of 1. They also saw that the students had a hard time figuring out how to loop through an arrays elements by using the elements index. The same problem was found by \textcite{KumarVeerasamy2016} that noticed that students often were of by one index when looping through the list using index, causing index-error when trying to extract an element with an index larger then the length of the list. They also saw a tendency to use negative index-numbers when unneccesary to extract elements from an array. 

When using data types in CS1 the students are often required to use some kind of comparison between variables. This creates situations where the students try to compare different data types to each other, which shows that the students are not completely familiar to the difference between data types and how they are initalized and later used \parencite{Kurvinen2016}. When objects are introduced to the students later in the course it will create more confusion when the students are suppose to compare objects with simple data types, which can be made easier if comparison between types are repeated by the teacher at the end of the course \parencite{Kurvinen2016}. 

