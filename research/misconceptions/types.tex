\subsection{Data types}

When introduced to programming students will often encounter several different data types, for example strings, integers, arrays and dictionaries. As excepted when learning about several data types during a short time-span different misconceptions occur. 

When using data types in CS1 the students are often required to use some kind of comparison between variables. This creates situations where the students try to compare different data types to each other, which shows that the students are not completely familiar to the difference between data types and how they are initialised and then later used \parencite{Kurvinen2016}. When objects are introduced to the students later in the course it will create more confusion when the students are suppose to compare objects with simple data types, which can be made easier if comparison between types are repeated by the teacher at the end of the course \parencite{Kurvinen2016}. XXX Do we think this is interesting enough to mention? Is it part of the scope?

XXX Add analysis on how to introduce different data types so that the students understand the difference between them


 \textcite{Kurvinen2016} found in their studies that it is not intuitive for students that the indexation of an array starts at 0 instead of 1. They also noticed that students have a hard time figuring out how to loop through an array's elements by using the elements index. The same problem was found by \textcite{KumarVeerasamy2016} who found that students often are of by one index when looping through an array using index, causing index-error when trying to extract an element with an index larger then the length of the list. They also saw a tendency to use negative index-numbers when unnecessary to extract elements from an array.

 XXX Insert analysis on how we can teach indexation of arrays


\textbf{Fun things to add:}

- "Omitting quotes for strings: In natural language, there is no need to use quotes for prose. Omitting a single-word string like foo(Alice) is legal code but accesses an undefined variable; omitting a multi-word string like ‘foo(My name is Alice)’ leads to a parse error. Escape sequences like \" are also confusing since they are not needed in English." \parencite{GuoMarkelZhang2020}

- "When numerical data is read from files or terminal input, they often start as strings. If they are not properly converted to numbers, it is still possible to use math operators like + and > on them, which will perform string concatenation and comparison, respectively. These can lead to subtle semantic and logic errors, even though the code does not crash" \parencite{GuoMarkelZhang2020}