\mode*
\subsection{Problem solving}

\subsubsection{Role in the syllabus}

\subsubsection{Difficulties that can occur}


Here I want to have some articles about how math-problems will make it harder 
for students to solve the problem. Quote from Veerasamy et al \emph{This 
study analysis also explored that novices of programming struggled in writing 
code for math-related Questions 6 and 7 (refer Table C1). Nearly 66\% of 
students did not do well in the mathematical problem-based questions though 
explained and allowed to surf the Internet to seek for more details during 
the exam hours. A neo-Piagetian theory of cognitive development stated that 
students who are at the concrete operational stage struggle to write large 
programs with partial specifications, although they can write small programs 
from well-defined specifications (Teague et al., 2012).}

Also I would want to include difficulties students have when debugging the 
code 
and trying to find errors. Students often have a problem with tracing the 
code, 
something that is discussed by \textcite[p.~20]{Sleeman1984}. On the same 
subject as above: In what order a program will be executed in, Programming 
misconceptions in an introductory level programming course exam by Einari 
Kurvinen, Niko Hellgren, Erkki Kaila, Mikko-Jussi Laakso, Tapio Salakoski

Would also maybe like to mention how a lab instruction should be to help 
students get the right knowledge from the lab. Is discussed somewhere in 
Yizhou Qian and James Lehmans article I think.

