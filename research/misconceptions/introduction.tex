\mode*
\section{Introduction}

When students are introduced to the world of programming, misconceptions are 
bound to happen.
We usually come across these misconceptions when tutoring students on 
assignments or labs.
Misconceptions occur when a student has discerned some aspects of a phenomenon,  
but not others \parencite{NCOL}.
This means that something was amiss in the teaching, or at least that the 
student didn't discern it.
The ultimate goal of our teaching is that the students fully master the 
educational objectives, without any misconceptions.

\begin{frame}<presentation>
  \begin{block}{Teaching programming}
    \begin{itemize}
      \item We teach students programming.
      \item The students learn some, but not all.
      \item Misconceptions occur.
      \item We fix the misconceptions during tutoring.
    \end{itemize}
  \end{block}

  \pause

  \begin{definition}[Misconception]
    \begin{itemize}
      \item A student has discerned some aspects of a phenomenon, but not 
        others~\parencite{NCOL}.
    \end{itemize}
  \end{definition}
\end{frame}

In this work, we use the literature documenting common misconceptions as the 
phenomenographic observations used in a variation theory analysis.
The goal is to derive patterns of variation to use when teaching introductory 
programming.

\begin{frame}<presentation>
  \begin{block}{Goal}
    \begin{itemize}
      \item Use literature of misconceptions as phenomenographic 
        observations.
      \item Derive patterns of variation to use when teaching introductory 
        programming.
    \end{itemize}
  \end{block}
\end{frame}

According to variation theory \parencite[Ch.~2]{NCOL}, each educational 
objective can be divided into different \emph{aspects}.
Consider the educational objective \enquote{the student should be able to use 
functions}.
One aspect of this particular educational objective is the local and global 
scope of variables.
Another aspect is returning values from a function.
For a student to achieve the educational objective, she must be able to discern 
\emph{all} the different aspects of the educational objective.
Aspects that the student hasn't yet discerned are critical aspects.
And when a critical aspect is not discerned, misconceptions can occur.

According to the theory, one necessary condition for learning is that the 
student is introduced to a series of patterns of variation in the dimension of 
each critical aspect.
Misconceptions are examples of when a student has failed to discern (at least) 
one critical aspect.
This allows us to use misconceptions to inform our designs when designing 
teaching according to variation theory.

\subsection{Research questions}

The purpose of this study is to answer the following questions:
\begin{frame}
\begin{enumerate}
  \item<1> What misconceptions in introductory programming has been identified 
    by earlier research?
  \item<2-3> Based on these misconceptions, what aspects (in terms of variation 
    theory) of introductory programming can we identify?
  \item<3> What patterns of variation can be used to discern these aspects, to 
    teach them to students effectively?
  \item<4> How do the patterns of variation relate to the skill of debugging?
  %\item<+> Where do we need further research?
\end{enumerate}
\end{frame}
