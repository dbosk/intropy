% !TEX TS-program = pdflatex
% !TEX encoding = UTF-8 Unicode

% This is a simple template for a LaTeX document using the "article" class.
% See "book", "report", "letter" for other types of document.

\documentclass[11pt]{article} % use larger type; default would be 10pt

\usepackage[utf8]{inputenc} % set input encoding (not needed with XeLaTeX)

%%% Examples of Article customizations
% These packages are optional, depending whether you want the features they provide.
% See the LaTeX Companion or other references for full information.

%%% PAGE DIMENSIONS
\usepackage{geometry} % to change the page dimensions
\geometry{a4paper} % or letterpaper (US) or a5paper or....
% \geometry{margin=2in} % for example, change the margins to 2 inches all round
% \geometry{landscape} % set up the page for landscape
%   read geometry.pdf for detailed page layout information

\usepackage{graphicx} % support the \includegraphics command and options

% \usepackage[parfill]{parskip} % Activate to begin paragraphs with an empty line rather than an indent

%%% PACKAGES
\usepackage{booktabs} % for much better looking tables
\usepackage{array} % for better arrays (eg matrices) in maths
\usepackage{paralist} % very flexible & customisable lists (eg. enumerate/itemize, etc.)
\usepackage{verbatim} % adds environment for commenting out blocks of text & for better verbatim
\usepackage{subfig} % make it possible to include more than one captioned figure/table in a single float
% These packages are all incorporated in the memoir class to one degree or another...

%%% HEADERS & FOOTERS
\usepackage{fancyhdr} % This should be set AFTER setting up the page geometry
\pagestyle{fancy} % options: empty , plain , fancy
\renewcommand{\headrulewidth}{0pt} % customise the layout...
\lhead{}\chead{}\rhead{}
\lfoot{}\cfoot{\thepage}\rfoot{}

%%% SECTION TITLE APPEARANCE
\usepackage{sectsty}
\allsectionsfont{\sffamily\mdseries\upshape} % (See the fntguide.pdf for font help)
% (This matches ConTeXt defaults)

%%% ToC (table of contents) APPEARANCE
\usepackage[nottoc,notlof,notlot]{tocbibind} % Put the bibliography in the ToC
\usepackage[titles,subfigure]{tocloft} % Alter the style of the Table of Contents
\renewcommand{\cftsecfont}{\rmfamily\mdseries\upshape}
\renewcommand{\cftsecpagefont}{\rmfamily\mdseries\upshape} % No bold!

%%% END Article customizations

%%% The "real" document content comes below...

\title{Teaching CS1}
\author{Celina Soori (at this moment}
%\date{} % Activate to display a given date or no date (if empty),
         % otherwise the current date is printed 

\begin{document}
\maketitle

\section{Introduction}

Introduce and give som background on the subject and why we have chosen to analyze this. 

\subsection{Purpose}

The purpose of this article is to give an overview of the studies that has been conducted on this subject and analyze these with help of variation theory to understand how one is supposed to teach CS1 to help students grasp important concepts in learning programming. Another purpose with this article is to discuss what further studies we need to conduct to understand deeper what students difficulties there are and how the education can be designed to avoid these. 

\subsection{Method}

Kanske ska vara en egen section.

\section{Variation Theory}

Gå igenom vad variationsteori är. 

\section{Important modules in teaching CS1}

This section is divided to reflect the different concepts that are teached during CS1. Each section will describe what students often are meant to learn and understand in that module, which is then followed with a summary of what different studies have found is difficult for students in that module. 

\subsection{Functions and variables}

\subsubsection{Role in the syllabus}

Functions and variables are typically learned at the same time because they go hand in hand osv....

Vill vi ha detta? Är det överflödigt? Kanske bra med en introduktion till vad modulen innehåller? Jag tror jag röstar ja!

\subsubsection{Difficulties that can occure}

According to Yizhou Qian and James Lehman (2017) in their article \emph{Students’ Misconceptions and Other Difficulties in Introductory Programming: A Literature Review} students have difficulties understanding how variables and that the students usually make assumptions about variables that are wrong. For example one assumption is that variables can hold more then one value at the time. According to other studies sometimes students even think that variables can hold an entire algorithm and therefor see a variable as a function or equation. This will create problems when a student creates a variable in belief that the variable will change its value when the equation is supposed to change its value or at the time when the variable is used in the programme (Kohn, 2017; Plass-Oude Bos, 2015). Kohn (2017) explains that this misconseption can be connected to how variable definitions are used in math. 

But it is not only the right side of the variable definition that students can have misconceptions about. The name of the variable has been misunderstood as having power of the value which it holds (Qian \& Lehman, 2017). For example if a student names one variable \emph{max} and another variable \emph{min}, the student might think that the variables will strictly only hold the maximum value and the minimum value througout the programme. 

If we move on to the relationship between variables and functions we can see more conceptual misconseptions that students have. The first difficulty is how students treat return-values. When a function is supposed to return a value some students miss the return value, expecting the function to return it by default (Kurvinen et al., 2016). Some students might also believe that a print-statement at the end of a function will act as a return statement (Qian \& Lehman, 2017). Another misconception that goes hand in hand with the assumption that a variable holds an equation and not a single value, is that if in the return statement the student returns an equation, the return value will be that equation, not the value that the equation represents (Kohn, 2017). 

The difference between variables in programming and variables in math is differences that some students does not grasp. If a student in a variable definition uses on the right side of the equal symbol a variable that is not defined, but the variable on the left side is already defined, they think that the computer will solve this as an equation (Plass-Oude Bos, 2015). This assumption by the students is also something discovered by Kohn (2017) when giving the students the definition \emph{x = x + 1}. If you look at this definition with a mathematical perspective you will see an unsolvable equation, which is also what some of the students saw. They did not see that the x to the left is the variable, and that the x to the right only holds a value. This definition is  easier to understand for a novice programmer according to Kohn (2017) when we instead write this definition as \emph{x += 1}. 


\subsection{Data types}

\subsubsection{Role in the syllabus}

\subsubsection{Difficulties that can occure}

Indexering i listor: Programming misconceptions in an introductory level programming course exam av Einari Kurvinen, Niko Hellgren, Erkki Kaila, Mikko-Jussi Laakso, Tapio Salakoski

Comparision between different types: Programming misconceptions in an introductory level programming course exam av Einari Kurvinen, Niko Hellgren, Erkki Kaila, Mikko-Jussi Laakso, Tapio Salakoski


\subsection{Classes and objects}

\subsubsection{Role in the syllabus}

\subsubsection{Difficulties that can occure}

\subsection{Repetitions}

\subsubsection{Role in the syllabus}

\subsubsection{Difficulties that can occure}

Variables in loops:Tracing quiz set to identify novices' programming misconceptions av Takayuki Sekiya, Kazunori Yamaguchi

Variables in if-statements that are in for-loops: Tracing quiz set to identify novices' programming misconceptions av Takayuki Sekiya, Kazunori Yamaguchi, Yizhou Qian and James Lehman (2017)

\subsection{Problemsolving}

Här vill jag ha med hur matte-uppgifter kan försvåra för eleverna att lösa problemet. Skulle även vilja ta upp hur man ska göra en labbeskrivning för att hjälpa studenterna att få ut det mesta från labben. 

\subsubsection{Role in the syllabus}

\subsubsection{Difficulties that can occure}

\subsection{Things that doesnt fit in anywhere else..}

Vilken ordning kod exekveras: Programming misconceptions in an introductory level programming course exam av Einari Kurvinen, Niko Hellgren, Erkki Kaila, Mikko-Jussi Laakso, Tapio Salakoski

If-statements: Om det genererar false kommer programmet att avslutas, om true kommer programmet exekveras. Plass-Oude Bos, 2015. 


\section{Analysis}

Analysen ska ske mha variationsteori. 

\section{Discussion}

\end{document}
