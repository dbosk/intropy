\mode*

\begin{frame}
  \url{https://github.com/dbosk/intropy/classes/slides-more/examples}
\end{frame}

\section{En klass för bråk}

\begin{frame}
  myfrac\textunderscore base.py
\end{frame}

\subsection{Konstrueraren}

\begin{frame}[fragile]
  \inputminted[linenos,firstline=3,lastline=13]{python}{examples/myfrac_base.py}

  \begin{example}
    \begin{minted}{python}
      frac = Fraction(1, 2)
      int_as_frac = Fraction(2)
      frac_from_frac = Fraction(Fraction(1, 2))
      half_frac = Fraction(Fraction(1, 2), 2)
    \end{minted}
  \end{example}
\end{frame}

\subsection{Attribut som egenskaper}

\begin{frame}[fragile]
  \inputminted[linenos,firstline=3,lastline=3]{python}{examples/myfrac_base.py}
  \dots
  \inputminted[linenos,autogobble=off,firstline=15,lastline=24]{python}{examples/myfrac_base.py}

  \begin{example}
    \begin{minted}{python}
      frac = Fraction(1, 2)
      print(f"{frac.nominator}/{frac.denominator}")
    \end{minted}
  \end{example}
\end{frame}

\begin{frame}[fragile]
  \begin{columns}[t]
    \begin{column}{0.5\columnwidth}
      \begin{minted}{python}
        class Class:
          @property
          def property(self):
            return self.__property

          @property.setter
          def property(self, value):
            self.__property = value

        obj = Class()
        print(obj.property)
        obj.property = new_value
      \end{minted}
    \end{column}
    \begin{column}{0.5\columnwidth}
      \begin{minted}{python}
        class Class:
          def get_property(self):
            return self.__property

          def set_property(self, value):
            self.__property = value

        obj = Class()
        print(obj.get_property())
        obj.set_property(new_value)
      \end{minted}
    \end{column}
  \end{columns}
\end{frame}

\subsection{Typkonvertering}

\begin{frame}[fragile]
  \inputminted[linenos,firstline=3,lastline=3]{python}{examples/myfrac_base.py}
  \dots
  \inputminted[linenos,autogobble=off,firstline=25,lastline=30]{python}{examples/myfrac_base.py}

  \begin{example}
    \begin{minted}{python}
      frac = Fraction(1, 2)
      print(frac)
      print(float(frac))
    \end{minted}
  \end{example}
\end{frame}


\section{Addition och subtraktion}

\subsection{Addition}

\begin{frame}
  myfrac\textunderscore add.py
\end{frame}

\begin{frame}[fragile]
  \inputminted[linenos,firstline=3,lastline=3]{python}{examples/myfrac_add.py}
  \dots
  \inputminted[autogobble=false,linenos,firstline=31,lastline=40]{python}{examples/myfrac_add.py}

  \begin{example}
    \begin{minted}{python}
      frac_a = Fraction(1, 2)
      frac_b = Fraction(1, 3)
      print(frac_a + frac_b)
      print(1 + frac_a) # funkar ej
    \end{minted}
  \end{example}
\end{frame}

\begin{frame}[fragile]
  \inputminted[linenos,firstline=3,lastline=3]{python}{examples/myfrac_add.py}
  \dots
  \inputminted[autogobble=false,linenos,firstline=41,lastline=43]{python}{examples/myfrac_radd.py}

  \begin{example}
    \begin{minted}{python}
      frac_a = Fraction(1, 2)
      print(1 + frac_a)
    \end{minted}
  \end{example}
\end{frame}

\subsection{Negation}

\begin{frame}
  myfrac\textunderscore sub.py
\end{frame}

\begin{frame}[fragile]
  \inputminted[linenos,firstline=3,lastline=3]{python}{examples/myfrac_sub.py}
  \dots
  \inputminted[autogobble=false,linenos,firstline=44,lastline=45]{python}{examples/myfrac_sub.py}

  \begin{example}
    \begin{minted}{python}
      frac_a = Fraction{1, 2}
      print(-frac_a)
    \end{minted}
  \end{example}
\end{frame}

\subsection{Subtraktion}

\begin{frame}[fragile]
  \inputminted[linenos,firstline=3,lastline=3]{python}{examples/myfrac_sub.py}
  \dots
  \inputminted[autogobble=false,linenos,firstline=47,lastline=48]{python}{examples/myfrac_sub.py}

  \begin{example}
    \begin{minted}{python}
      frac_a = Fraction{1, 2}
      frac_b = Fraction{1, 4}
      print(frac_a - frac_b)
      print(1 - frac_a) # funkar ej
    \end{minted}
  \end{example}
\end{frame}

\begin{frame}[fragile]
  \inputminted[linenos,firstline=3,lastline=3]{python}{examples/myfrac_sub.py}
  \dots
  \inputminted[autogobble=false,linenos,firstline=50,lastline=51]{python}{examples/myfrac_sub.py}

  \begin{example}
    \begin{minted}{python}
      frac_a = Fraction{1, 2}
      print(1 - frac_a)
    \end{minted}
  \end{example}
\end{frame}


\section{Multiplikation}

\begin{frame}[fragile]
  \inputminted[linenos,firstline=3,lastline=3]{python}{examples/myfrac_mul.py}
  \dots
  \inputminted[autogobble=false,linenos,firstline=53,lastline=59]{python}{examples/myfrac_mul.py}

  \begin{example}
    \begin{minted}{python}
      frac_a = Fraction{1, 2}
      frac_b = Fraction{1, 3}
      print(frac_a * frac_b)
    \end{minted}
  \end{example}
\end{frame}

\begin{frame}[fragile]
  \inputminted[linenos,firstline=3,lastline=3]{python}{examples/myfrac_mul.py}
  \dots
  \inputminted[autogobble=false,linenos,firstline=61,lastline=62]{python}{examples/myfrac_mul.py}

  \begin{example}
    \begin{minted}{python}
      frac_a = Fraction{1, 2}
      print(frac_a * 2)
      print(2 * frac_a)
    \end{minted}
  \end{example}
\end{frame}


\section{Förkortning}

\begin{frame}
  \begin{exercise}
    \begin{itemize}
      \item Hur kan vi implementera förkortning av bråk?
      \item Exempelvis \(\frac{2}{6}\) bör ju skrivas \(\frac{1}{3}\).
    \end{itemize}
  \end{exercise}
\end{frame}
