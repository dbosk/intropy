\title{%
  Algoritmiskt tänkande
}
\author{Daniel Bosk}
\institute{%
  KTH EECS
}

\mode<article>{\maketitle}
\mode<presentation>{%
  \begin{frame}
    \maketitle
  \end{frame}
}

\mode*

\begin{abstract}
  % What's the problem?
% Why is it a problem? Research gap left by other approaches?
% Why is it important? Why care?
% What's the approach? How to solve the problem?
% What's the findings? How was it evaluated, what are the results, limitations, 
% what remains to be done?

% XXX Summary
\emph{Summary:}
\dots

% XXX Motivation and intended learning outcomes
\emph{Intended learning outcomes:}
\dots

% XXX Prerequisites
\emph{Prerequisites:}
\dots

% XXX Reading material
\emph{Reading:}
\dots

\end{abstract}


\section{Algoritmer}

% diska
% sortera tvätt innan tvätt
% hänga tvätt
% vika tvätt

\begin{frame}
  \begin{example}[Att göra pannkakssmet]
    \begin{enumerate}
      \item Knäck tre ägg i en bunke.
      \item Vispa ordentligt.
      \item Häll i \SI{3}{\deci\litre} mjölk.
      \item Vispa ordentligt.
      \item \label{mjöl} Häll i \SI{1}{\deci\litre} mjöl.
      \item \label{vispa-mjöl} Vispa ordentligt.
      \item Repetera \ref{mjöl} och \ref{vispa-mjöl} två gånger till.
      \item Häll i \SI{1/2}{tsk} salt.
      \item Häll i \SI{2}{msk} smält smör.
      \item Vispa ordentligt.
    \end{enumerate}
  \end{example}
\end{frame}

\begin{frame}[fragile]
  \begin{definition}[Algoritm]
    \begin{itemize}
      \item Algoritm, algorism: ursprungligen förvanskning av namnet på den 
        persisk-arabiska matematikern al-Khwarizmi\footnote{%
          \emph{Nationalencyklopedin},
          algoritm. 
          \url{http://www-ne-se.focus.lib.kth.se/uppslagsverk/encyklopedi/lång/algoritm}
          (hämtad 2022-09-04).
        }
      \item Definition: \enquote{inom matematik och databehandling en 
          systematisk procedur som i ett ändligt antal steg anger hur man utför 
        en beräkning eller löser ett givet problem}.
    \end{itemize}
  \end{definition}
\end{frame}

\begin{frame}
  \begin{exercise}[Sortera tvätt]
    \begin{itemize}
      \item Beskriv din algoritm för att sortera tvätt innan den åker in i 
        tvättmaskinen.
    \end{itemize}
  \end{exercise}
\end{frame}

\begin{frame}[fragile]
  \begin{example}[Sortera tvätt]
    \begin{enumerate}
      \item \label{ta-från-tvättkorg} Ta \alert<3>{en textil} ur tvättkorgen.
      \item Om \alert<3>{textilen} är vit, lägg i vit 60; annars
      \item Om \alert<3>{textilen} är mörk och underkläder, lägg i mörk 60; 
        annars
      \item Om \alert<3>{textilen} är mörk, lägg i mörk 40; annars
      \item Om \alert<3>{textilen} är ljus och underkläder, lägg i ljus 60; 
        annars
      \item Om \alert<3>{textilen} är ljus, lägg i ljus 40; annars
        \item Lägg \alert<3>{textilen} i hantera senare-högen.
      \item Repetera från \ref{ta-från-tvättkorg}.
    \end{enumerate}
  \end{example}

  \begin{onlyenv}<2>
    \begin{exercise}
      \begin{itemize}
        \item Vad varierar i exemplet med tvättsortering?
      \end{itemize}
    \end{exercise}
  \end{onlyenv}
\end{frame}

\subsection{Variabler}

\begin{frame}
  \begin{exercise}
    \begin{itemize}
      \item Vad varierar i exemplet med pannkakssmet?
    \end{itemize}
  \end{exercise}

  \begin{example}[Att göra pannkakssmet]
    \begin{enumerate}
      \item Knäck tre ägg i en bunke.
      \item Vispa ordentligt.
      \item Häll i \SI{3}{\deci\litre} mjölk.
      \item Vispa ordentligt.
      \item \label{mjöl} Häll i \SI{1}{\deci\litre} mjöl.
      \item \label{vispa-mjöl} Vispa ordentligt.
      \item Repetera \ref{mjöl} och \ref{vispa-mjöl} två gånger till.
      \item Häll i \SI{1/2}{tsk} salt.
      \item Häll i \SI{2}{msk} smält smör.
      \item Vispa ordentligt.
    \end{enumerate}
  \end{example}
\end{frame}

\begin{frame}
  \begin{example}[Att göra pannkakssmet för \(n\) personer]
    \begin{enumerate}
      \item Knäck \(\frac{3n}{4}\) ägg i en bunke.
      \item Vispa ordentligt.
      \item Häll i \SI{\frac{3n}{4}}{\deci\litre} mjölk.
      \item Vispa ordentligt.
      \item För totalt \SI{\frac{3n}{4}}{\deci\litre} mjöl:
        \begin{enumerate}
          \item Häll i \SI{1}{\deci\litre} mjöl.
          \item Vispa ordentligt.
        \end{enumerate}
      \item Häll i \SI{\frac{3n}{2\cdot 4}}{tsk} salt.
      \item Häll i \SI{\frac{2n}{4}}{msk} smält smör.
      \item Vispa ordentligt.
    \end{enumerate}
  \end{example}
\end{frame}

\begin{frame}[fragile]
  \begin{definition}[Variabel]
    \begin{itemize}
      \item Något som hänvisar till något som kan variera kallar vi för 
        \emph{variabel}.
    \end{itemize}
  \end{definition}

  \pause

  \begin{example}[Variabler]
    \begin{itemize}
      \item Textilen
      \item Antal personer, \(n\).
    \end{itemize}
  \end{example}
\end{frame}

\subsection{(Del)algoritmer, eller funktioner}

\begin{frame}
  \begin{example}[Att steka pannkakor\only<2>{ \alert{för fyra personer}}]
    \begin{enumerate}
      \item \alert<2>{Gör pannkakssmet för fyra personer.}
      \item För varje portion, medan det finns pannkakssmet kvar:
        \begin{enumerate}
          \item Häll \SI{1}{\deci\litre} smet i en het stekpanna.
          \item Vänta \SI{2}{\minute}.
          \item Vänd pannkakan.
          \item Vänta \SI{2}{\minute}.
          \item Servera pannkakan.
        \end{enumerate}
    \end{enumerate}
  \end{example}

  \begin{onlyenv}<1>
    \begin{exercise}
      \begin{itemize}
        \item Det resterande, som inte är en variabel, vad är det?
      \end{itemize}
    \end{exercise}
  \end{onlyenv}
  
  \begin{onlyenv}<2>
    \begin{definition}[Funktion]
      \begin{itemize}
        \item En funktion är en fixerad procedur.
        \item Den tar variabler som inmatning och returnerar någon 
          utmatning.
      \end{itemize}
    \end{definition}
  \end{onlyenv}
\end{frame}

\section{Uppdelning i delalgoritmer (funktioner)}

\begin{frame}[fragile]
  \begin{exercise}
    \begin{itemize}
      \item Vilka funktioner kan ni hitta i funktionen?
    \end{itemize}
  \end{exercise}

  \begin{example}[Att göra pannkakssmet för \(n\) personer]
    \begin{enumerate}
      \item \alert<2>{Knäck} \(\frac{3n}{4}\) ägg i en bunke.
      \item \alert<2>{Vispa} ordentligt.
      \item \alert<2>{Häll} i \SI{\frac{3n}{4}}{\deci\litre} mjölk.
      \item \alert<2>{Vispa} ordentligt.
      \item För totalt \SI{\frac{3n}{4}}{\deci\litre} mjöl:
        \begin{enumerate}
          \item \alert<2>{Häll} i \SI{1}{\deci\litre} mjöl.
          \item \alert<2>{Vispa} ordentligt.
        \end{enumerate}
      \item \alert<2>{Häll} i \SI{\frac{3n}{2\cdot 4}}{tsk} salt.
      \item \alert<2>{Häll} i \SI{\frac{2n}{4}}{msk} smält smör.
      \item \alert<2>{Vispa} ordentligt.
    \end{enumerate}
  \end{example}
\end{frame}

\begin{frame}
  \begin{remark}
    \begin{itemize}
      \item Funktioner motsvarar verb i dagligt språk!
    \end{itemize}
  \end{remark}
\end{frame}

\begin{frame}
  \begin{exercise}
    \begin{itemize}
      \item Vilka funktioner och variabler använder du i din algoritm för 
        handdisk i köket?
    \end{itemize}
  \end{exercise}
\end{frame}

\begin{frame}
  \begin{example}[Algoritm för handdisk]
    \begin{enumerate}
      \item \alert{Lägg} \emph{all disk} i diskhon med diskvattnet i.
      \item \label{glas} \alert{Ta} \emph{ett glas}, \alert{diska} \emph{det}, 
        \alert{lägg} \emph{det} i diskhon för diskad disk.
      \item Repetera \ref{glas} tills att \emph{glasen} är slut.
      \item \label{bestick} \alert{Ta} \emph{ett bestick}, \alert{diska} 
        \emph{det}, \alert{lägg} \emph{det} i diskhon för diskad disk.
      \item Repetera \ref{bestick} tills att besticken är slut.
      \item \label{tallrik} \alert{Ta} \emph{en tallrik}, \alert{diska} 
        \emph{den}, \alert{lägg} \emph{den} i diskhon för diskad disk.
      \item Repetera \ref{tallrik} tills att tallrikarna är slut.
    \end{enumerate}
  \end{example}
\end{frame}

