\mode*

\section{Tupler}

\begin{frame}[fragile]
  \begin{example}[tuples.py]
    \inputminted{python}{examples/tuples.py}
  \end{example}

  \pause

  \begin{example}
    \begin{minted}{python}
def my_enumerate(iterable):
  """Returnerar en lista med (n, item) där item är element i iterable."""
  result = []

  for i in range(len(iterable)):
    result.append((i, iterable[i]))

  return result
    \end{minted}
  \end{example}
\end{frame}

\begin{frame}[fragile]
  \begin{example}[fullname.py]
    \inputminted[firstline=3,lastline=7]{python}{examples/fullname.py}
  \end{example}

  \pause

  \begin{example}[fullname-alt.py]
    \inputminted[firstline=3,lastline=7,highlightlines=7]{python}{examples/fullname-alt.py}
  \end{example}
\end{frame}

\begin{frame}[fragile]
  \begin{exercise}
    \begin{itemize}
      \item Vad är skillnaden mellan en lista och en tupel?
      \item \mintinline{python}|a = [1, 2, 3]|
      \item \mintinline{python}|b = (1, 2, 3)|
    \end{itemize}
  \end{exercise}
\end{frame}

