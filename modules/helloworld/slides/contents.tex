\title{%
  Hello World!
}
\author{Daniel Bosk}
\institute{%
  KTH EECS
}

\mode<article>{\maketitle}
\mode<presentation>{%
  \begin{frame}
    \maketitle
  \end{frame}
}

\mode*

\begin{abstract}
  According to variation theory (see~\cite{NCOL}\footfullcite{NCOL}), learning an 
educational objective requires the learner to distinguish all aspects of that 
educational objective. What these aspects are is hard to tell for someone who 
has already mastered the educational objective in question. However, 
misconceptions occur when the learner cannot yet distinguish one or more 
critical aspects. Thus misconceptions can help us identify what those 
(critical) aspects are. Then we can teach the learner to distinguish those 
critical aspects by varying examples through a series of patterns of variation.

In our work (in progress), we explore the existing literature on misconceptions 
in introductory programming courses and analyse it through the lens of 
variation theory to identify the necessary aspects of programming that a 
learner must learn to distinguish. We also outline patterns of variation to 
teach to distinguish these aspects.

\Textcite{NCOL} also connects the patterns of variation of variation theory to 
both deep learning and scientific discoveries. In both cases, the learners (the 
researcher is also a learner of the unknown) introduce variation for themselves 
through these patterns of variation. We hypothesize the connection between the 
patterns of variation and the skill of debugging (the programmer is learning 
about some unknown when debugging).

\end{abstract}


\section{Programmering}

\subsection{Vad är programmering?}

\begin{frame}
  \begin{block}{Dator}
    \begin{itemize}
      \item Processor
      \item Minne
      \item Annan hårdvara
    \end{itemize}
  \end{block}
\end{frame}

\begin{frame}
  \begin{block}{Processorn}
    \begin{itemize}
      \item Exekverar maskinkod --- program
    \end{itemize}
  \end{block}

  \pause

  \begin{example}
    \begin{itemize}
      \item<1-2> 00012000050000020500
      \item<3-> 0001 2000 0500
      \item<3-> 0002 0500
    \end{itemize}
  \end{example}
\end{frame}

\begin{frame}
  \begin{block}{Svar}
    \begin{itemize}
      \item Programmering är att skriva program.
    \end{itemize}
  \end{block}
\end{frame}


\subsection{Programmeringsspråk}

\begin{frame}
  \begin{block}{Programmeringsspråk}
    \begin{itemize}
      \item Ett språk för människor att skriva datorprogram.
    \end{itemize}
  \end{block}
\end{frame}

\begin{frame}[fragile]
  \begin{example}[Python]
    \begin{minted}{python}
print("Hello, World!")
    \end{minted}
  \end{example}

  \pause

  \begin{example}[C++]
    \begin{minted}{c++}
#include <iostream>
using namespace std;

int main(void) {
    std::cout << "Hello, World!" << std::endl;
    return 0;
}
    \end{minted}
  \end{example}
\end{frame}

\begin{frame}
  \begin{remark}
    \begin{itemize}
      \item C++ måste översättas till maskinkod!
      \item Python läses av ett program (maskinkod), och det programmet 
        exekverar koden.
    \end{itemize}
  \end{remark}
\end{frame}

\begin{frame}[fragile]
  \begin{example}[Terminalen]
    \begin{minted}{bash}
echo "Hello World!"
    \end{minted}
  \end{example}

  \pause

  \begin{remark}
    \begin{itemize}
      \item Bash läser varje rad och exekverar olika program.
      \item Dessa program är redan konverterade till maskinkod.
    \end{itemize}
  \end{remark}
\end{frame}


\section{Köra Python}

\subsection{Terminalen}

\begin{frame}
  \begin{figure}
    \centering
    \includegraphics[height=0.8\textheight]{figs/python-terminal.png}
    \caption{Ett kodexempel i Python som skriver ut \enquote{hello world!} i 
    terminalen.}
  \end{figure}
\end{frame}


\subsection{Andra gränssnitt}

\begin{frame}
  \begin{figure}
    \centering
    \includegraphics[height=0.8\textheight]{figs/idle.jpg}
    \caption{Kod och körning av pythonkod i Idle.}
  \end{figure}
\end{frame}

\begin{frame}
  \begin{figure}
    \centering
    \includegraphics[height=0.8\textheight]{figs/codium.png}
    \caption{Gränssnittet till VSCodium (öppen källkodsvariant av Visual Studio 
    Code).}
  \end{figure}
\end{frame}

\begin{frame}
  \begin{figure}
    \centering
    \includegraphics[height=0.8\textheight]{figs/atom.png}
    \caption{Gränssnittet för Atom.}
  \end{figure}
\end{frame}

\begin{frame}
  \begin{figure}
    \centering
    \includegraphics[height=0.8\textheight]{figs/pycharm.png}
    \caption{Gränssnittet för PyCharm.}
  \end{figure}
\end{frame}

