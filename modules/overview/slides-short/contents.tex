\title{%
  Programmeringsteknik
}
\author{Daniel Bosk}
\institute{%
  KTH EECS
}

\mode<article>{\maketitle}
\mode<presentation>{%
  \begin{frame}
    \maketitle
  \end{frame}
}

\mode*

\begin{abstract}
  According to variation theory (see~\cite{NCOL}\footfullcite{NCOL}), learning an 
educational objective requires the learner to distinguish all aspects of that 
educational objective. What these aspects are is hard to tell for someone who 
has already mastered the educational objective in question. However, 
misconceptions occur when the learner cannot yet distinguish one or more 
critical aspects. Thus misconceptions can help us identify what those 
(critical) aspects are. Then we can teach the learner to distinguish those 
critical aspects by varying examples through a series of patterns of variation.

In our work (in progress), we explore the existing literature on misconceptions 
in introductory programming courses and analyse it through the lens of 
variation theory to identify the necessary aspects of programming that a 
learner must learn to distinguish. We also outline patterns of variation to 
teach to distinguish these aspects.

\Textcite{NCOL} also connects the patterns of variation of variation theory to 
both deep learning and scientific discoveries. In both cases, the learners (the 
researcher is also a learner of the unknown) introduce variation for themselves 
through these patterns of variation. We hypothesize the connection between the 
patterns of variation and the skill of debugging (the programmer is learning 
about some unknown when debugging).

\end{abstract}


\section{Programmering}

\begin{frame}[fragile]
  \begin{lstlisting}[language=python]
import math

def is_prime(n):
  """Checks if n is a prime, returns False if not a prime"""

  # We only need to check factors until $\(\lfloor\sqrt{n}\rfloor\)$
  for factor in range(2, math.floor(math.sqrt(n))):
    # If a factor divides $\(n\)$ evenly, $\(n\)$ is not a prime
    if n % factor == 0:
      return False

  return True
  \end{lstlisting}
\end{frame}

\section{Om kursen}

\subsection{Vad ska ni få ut?}

\begin{frame}
  \begin{block}<+>{Färdigheter}
    \begin{itemize}
      \item Algoritmiskt tänkande
      \item Problemlösningsförmåga
    \end{itemize}
  \end{block}

  \begin{block}<+>{Kunskaper}
    \begin{itemize}
      \item Terminologi
      \item Mental modell av datorsystem
    \end{itemize}
  \end{block}

  \begin{block}<+>{Verktyg}
    \begin{itemize}
      \item Python
    \end{itemize}
  \end{block}
\end{frame}

\subsection{Hur ska ni få ut detta?}

\begin{frame}
  \begin{block}<+>{Moduler}
    \begin{enumerate}
      \item Förberedelse (\enquote{Föreläsning})
      \item Övning
      \item Laboration
      \item Fördjupande övning
    \end{enumerate}
  \end{block}

  \begin{block}<+>{Projekt}
    \begin{itemize}
      \item Större uppgift.
      \item Andra halvan av kursen.
    \end{itemize}
  \end{block}
\end{frame}

\begin{frame}
  \begin{block}{Förberedelse (\enquote{Föreläsning})}
    \begin{itemize}
      \item Egna studier: OLI och FeedbackFruits.
      \item Interaktionen bra för lärande.
    \end{itemize}
  \end{block}

  \pause
  
  \begin{block}{Övning}
    \begin{itemize}
      \item Ger en översiktlig genomgång.
      \item Fokuserar på de viktigaste delarna.
      \item Får information från interaktionen under era förberedelser.
      \item Praktisk problemlösning för er!
    \end{itemize}
  \end{block}
\end{frame}

\begin{frame}
  \begin{block}{Laborationer}
    \begin{itemize}
      \item Driver ert lärande framåt.
      \item Lärande är socialt, arbeta i grupper om två.
    \end{itemize}
  \end{block}
\end{frame}

\begin{frame}[fragile]
  \begin{block}{Fördjupande övningar}
    \begin{itemize}
      \item Fördjupar era kunskaper och färdigheter från veckan.
      \item Förbereder för nästa vecka.
      \item Är även intressant för de som redan kan programmera.
    \end{itemize}
  \end{block}
\end{frame}

\section{Övrigt}

\begin{frame}
  \begin{block}<+>{Hur studerar ni?}
    \begin{itemize}
      \item På campus eller online.
      \item Allt kan göras online.
    \end{itemize}
  \end{block}

  \begin{block}<+>{Kursmaterial}
    \begin{itemize}
      \item Allt finns i Canvas.
    \end{itemize}
  \end{block}
\end{frame}

\begin{frame}[fragile]
  \begin{center}
    Mot Canvas \dots
  \end{center}
\end{frame}
