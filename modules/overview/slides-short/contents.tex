\title{%
  Programmeringsteknik
}
\author{Daniel Bosk}
\institute{%
  KTH EECS
}

\mode<article>{\maketitle}
\mode<presentation>{%
  \begin{frame}
    \maketitle
  \end{frame}
}

\mode*

\begin{abstract}
  % What's the problem?
% Why is it a problem? Research gap left by other approaches?
% Why is it important? Why care?
% What's the approach? How to solve the problem?
% What's the findings? How was it evaluated, what are the results, limitations, 
% what remains to be done?

% XXX Summary
\emph{Summary:}
\dots

% XXX Motivation and intended learning outcomes
\emph{Intended learning outcomes:}
\dots

% XXX Prerequisites
\emph{Prerequisites:}
\dots

% XXX Reading material
\emph{Reading:}
\dots

\end{abstract}


\section{Programmering}

\begin{frame}[fragile]
  \begin{lstlisting}[language=python]
import math

def is_prime(n):
  """Checks if n is a prime, returns False if not a prime"""

  # We only need to check factors until $\(\lfloor\sqrt{n}\rfloor\)$
  for factor in range(2, math.floor(math.sqrt(n))):
    # If a factor divides $\(n\)$ evenly, $\(n\)$ is not a prime
    if n % factor == 0:
      return False

  return True
  \end{lstlisting}
\end{frame}

\section{Om kursen}

\subsection{Vad ska ni få ut?}

\begin{frame}
  \begin{block}<+>{Färdigheter}
    \begin{itemize}
      \item Algoritmiskt tänkande
      \item Problemlösningsförmåga
    \end{itemize}
  \end{block}

  \begin{block}<+>{Kunskaper}
    \begin{itemize}
      \item Terminologi
      \item Mental modell av datorsystem
    \end{itemize}
  \end{block}

  \begin{block}<+>{Verktyg}
    \begin{itemize}
      \item Python
    \end{itemize}
  \end{block}
\end{frame}

\subsection{Hur ska ni få ut detta?}

\begin{frame}
  \begin{block}<+>{Moduler}
    \begin{enumerate}
      \item Förberedelse (\enquote{Föreläsning})
      \item Övning
      \item Laboration
      \item Fördjupande övning
    \end{enumerate}
  \end{block}

  \begin{block}<+>{Projekt}
    \begin{itemize}
      \item Större uppgift.
      \item Andra halvan av kursen.
    \end{itemize}
  \end{block}
\end{frame}

\begin{frame}
  \begin{block}{Förberedelse (\enquote{Föreläsning})}
    \begin{itemize}
      \item Egna studier: OLI och FeedbackFruits.
      \item Interaktionen bra för lärande.
    \end{itemize}
  \end{block}

  \pause
  
  \begin{block}{Övning}
    \begin{itemize}
      \item Ger en översiktlig genomgång.
      \item Fokuserar på de viktigaste delarna.
      \item Får information från interaktionen under era förberedelser.
      \item Praktisk problemlösning för er!
    \end{itemize}
  \end{block}
\end{frame}

\begin{frame}
  \begin{block}{Laborationer}
    \begin{itemize}
      \item Driver ert lärande framåt.
      \item Lärande är socialt, arbeta i grupper om två.
    \end{itemize}
  \end{block}
\end{frame}

\begin{frame}[fragile]
  \begin{block}{Fördjupande övningar}
    \begin{itemize}
      \item Fördjupar era kunskaper och färdigheter från veckan.
      \item Förbereder för nästa vecka.
      \item Är även intressant för de som redan kan programmera.
    \end{itemize}
  \end{block}
\end{frame}

\section{Övrigt}

\begin{frame}
  \begin{block}<+>{Hur studerar ni?}
    \begin{itemize}
      \item På campus eller online.
      \item Allt kan göras online.
    \end{itemize}
  \end{block}

  \begin{block}<+>{Kursmaterial}
    \begin{itemize}
      \item Allt finns i Canvas.
    \end{itemize}
  \end{block}
\end{frame}

\begin{frame}[fragile]
  \begin{center}
    Mot Canvas \dots
  \end{center}
\end{frame}
