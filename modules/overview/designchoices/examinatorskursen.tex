\section{Utvalda reflektioner från examinatorskursen}%
\label{examinatorskursen}

Detta är delar av mina reflektioner från examinatorskursen.

\subsection{Målrelaterad bedömning och validitet}%
\label{målrelaterad-bedömning-validitet}

\paragraph{Validitet i examination}

Det exempel som ges är bra, men det skulle
behöva kompletteras av ytterligare ett exempel. Det behövs ett exempel
på låg validitet (för att ge den nödvändiga kontrasten enligt
variationsteorin). Majoriteten av tentor som ges på KTH har med stor
sannolikhet låg validitet, då många är utformade på följande sätt. Vi
ger basfallet med två lärandemål och resten följer av induktion.

Lärandemål:
\begin{itemize}
\item
  Efter godkänd kurs ska studenten kunna redogöra för \(X\).
\item
  Efter godkänd kurs ska studenten kunna analysera \(Y\).
\end{itemize}

Examination:
\begin{itemize}
\item
  Tio frågor på ett prov, utformade så att måluppfyllelse kan visas.
\item
  Fem täcker \(X\) och fem täcker \(Y\).
\end{itemize}

Operationellt bedömningskriterium:
\begin{itemize}
\item
  Minst 6 av dessa 10 frågor ska vara korrekt besvarade för godkänt.
\end{itemize}

Denna utformning ger låg validitet då en student kan besvara fem frågor
om \(X\) korrekt, men bara en fråga om \(Y\); och trots detta får godkänt. 
Genom induktion får vi att detsamma gäller för N lärandemål. Således ger
tentor som slentrianmässigt har 50\% av poängen för E låg validitet.

\paragraph{Bonuspoäng}

Låt oss även utforska tillämpningen av bonuspoäng. Om bonuspoängen ges
som extra poäng på provet, godtyckligt, utan att motsvara en specifik
fråga (dvs ett specifikt lärandemål), då förstör även detta validiteten
i examinationen, på samma sätt som ovan. För att bonuspoängen inte ska
degradera validiteten måste varje bonuspoäng motsvara en specifik fråga
som motsvarar ett specifikt lärandemål. Uppgiften som ger bonus måste
också examinera just det lärandemålet. Dvs ett korrekt utformat
bonussystem flyttar (delar av) examinationen av lärandemål från ett
tillfälle till ett annat.

\subsection{Rättssäker examination}\label{rättssäker-examination}

\subsubsection{Proportionalitetsprincipen}\label{proportionalitetsprincipen}

Kan vi motivera användningen av bonuspoäng?

\begin{itemize}
\item
  Behövs åtgärden för att myndigheten ska uppnå de mål som gäller för
  den?

  \begin{itemize}
  \item
    Nej, vi kan examinera ändå.
  \item
    Ja, för vi vill ha genomströmning.
  \end{itemize}
\item
  Finns det mindre ingripande alternativ som ger samma eller nästan
  samma resultat?

  \begin{itemize}
  \item
    Ja, vi kan lära studenterna att planera sin tid, vilket dessutom
    borde vara mer värdefullt.
  \end{itemize}
\item
  Är skadorna som uppstår rimliga i förhållande till det mål som uppnås?

  \begin{itemize}
  \item
    Nej, det orsakar onödig stress och förstör möjligheten att planera
    för studenterna när vi har överdrivet mycket examination och det är
    examination hela tiden som ska vara klar just då för att få
    bonuspoäng. (Istället för att vi låter dem ansvara själv för sitt
    lärande.)
  \end{itemize}
\end{itemize}

Vi kan även fråga oss om vi behöver all den redundanta examination som
vi har i programmeringskurserna (labbar, datorprov och p-uppgift) och
andra kurser.

Vi återkommer till detta.

\subsubsection{Likhetsprincipen}\label{likhetsprincipen}

Är det motiverat att tillämpa bonuspoäng sett ur likhetsprincipens
perspektiv? (Denna diskussion är även högst relevant för området
Likvärdig examination, \cref{likvärdig-examination}, det området borde vara en 
konsekvens av denna princip.) Det är lättare för en barnlös 20-åring att hinna 
med alla deadlines skapade av bonuspoäng jämfört med en ensamstående
småbarnsmamma.

Rent teoretiskt borde inte bonuspoängen i sig inte ge några fördelar ur
examinationens perspektiv (\cref{målrelaterad-bedömning-validitet}). Detta är 
dock inte sant, men vi återkommer
till det. Låt oss anta att det är sant, då borde den ensamstående
småbarnsmamman kunna få samma betyg som den barnlösa 20-åringen utan
skillnad i prestation. Men då kan vi fråga varför vi har bonuspoängen
överhuvudtaget?

Om man tillämpar bonuspoängen på ett korrekt sätt så innebär de bara att
man flyttar examination (eller delar av) från ett tillfälle till ett
annat: I fallet med ett prov (eller tenta, som är vanligast att använda
bonuspoäng till) så flyttar man examinationen av en fråga (lärandemål)
från provet till tillfället då studenten kunde tillgodogöra sig
bonuspoängen.

Till och med här, i det optimala fallet, har den ensamstående
småbarnsmamman missgynnats med \emph{mindre tid per fråga} på provet,
jämfört med den barnlösa 20-åringen. Detta då tiden för provet är
konstant i praktiken, medan antalet frågor varierar på grund av
bonuspoängen. Vi kan även ta upp påverkan av andra faktorer som stress,
även om de är svårare att kvantifiera.

Låt oss nu titta på en mindre optimal tillämpning av bonuspoäng som är
mer lik hur bonuspoäng tillämpas i praktiken i många kurser.

Bonuspoängen ges som extra poäng på provet, godtyckligt, utan att
motsvara specifika lärandemål. Som sagt ovan 
(\cref{målrelaterad-bedömning-validitet}), förstör detta validiteten i 
examinationen.
Detta får ytterligare effekter sett ur likhetsperspektiv, då
bonussystemet nu inte längre flyttar examinationen från ett tillfälle
till ett annat. Att hinna i tid kan ge fördelar i helt andra delar av
examinationen. Det kan dessutom göra så att den barnlösa 20-åringen kan
få poäng för samma kunskaper/färdigheter två gånger (om studenten gör
frågan som motsvarar bonuspoängen igen på tentan utan att detta dras
av), vilket kan påverka betyget till det högre.

Sedan kan vi ju återvända till proportionalitetsprincipen för att
diskutera om fördelarna för de barnlösa 20-åringarna väger över
nackdelarna för de ensamstående småbarnsföräldrarna. Med tanke på att
den sociala snedrekryteringen till högskolan är ett genomgående problem,
skulle jag säga att det inte är värt det.

\subsection{Likvärdig examination}\label{likvärdig-examination}

\paragraph{Diskriminering}

Det som jag finner intressant är att fokus här väldigt
lätt låses till de olika diskrimineringsgrunderna. I min värld är
diskrimineringsgrunderna lågvattenmärket vi behöver hålla, helst bör vi
göra bättre än bara de sex diskrimineringsgrunderna. Exempelvis, den
sociala snedrekryteringen är inte diskriminering, men vi bör ändå göra
något åt det. Det är lite som med fokuset på mansdominansen inom
akademin/teknikområdet/etc., vilket leder fokuset till män och kvinnor,
men helt glömmer bort icke-binära och könsidentitet generellt. (En
anledning är väl att det inte går att mäta, då vi inte får fråga om det
datat; medan personnumret kan användas för att läsa ut man eller
kvinna.)

Diskrimineringsgrunderna måste man ta hänsyn till för att få ett E. Men
vi bör ha ambitionen att hamna på A.

\subsubsection{Strategier för likvärdig examination}%
\label{strategier-likvärdig-examination}

En väldigt bra sida, men \ldots{}

``Vi kan \emph{lära mer om} olika studentgrupper för att bättre lära
känna våra studenter och utforma examinationen inkluderande, vi kan
anpassa examinationen för olika studentgrupper'', men det blir svårt att
lära känna de studenter som är exkluderade på grund av
utbildningens/undervisningens generella utformande (campus kl 8--17 med
obligatorisk närvaro).

Policyn att deadline alltid ska vara kl 19 på vardagar är ett intressant
steg. Dock så vet jag inte om det hjälper småbarnsföräldrar så som det
är avsett. I min erfarenhet är det flexibiliteten som underlättar för
dem, rigida strukturer med tighta deadlines är svåra oavsett när på
dygnet de är.

En sak vi behöver göra på detta område är att se över hur mycket tid
våra kurser faktiskt tar i anspråk. Finns det faktiskt tid över till att
göra uppgifter utanför schemalagd tid? Vissa program är i vissa perioder
schemalagda kl 8--17, varje dag---då finns inte tid att göra något på
kursen när man inte är schemalagd.

I mitt eget fall schemalade jag egen studietid för att skydda
studenterna från andra kurser. Visade sig att de ändå bara hade tid för
två timmar utanför kursens ordinarie schemalagda tillfällen (en
föreläsning, en labb och en övning).

\subsubsection{Inkluderande examination}\label{inkluderande-examination}

``Ett annat exempel på inkluderande examination är att ge gott om tid
för inlämningsuppgifter, alltså från det att uppgiften publiceras (givet
att studenterna då har nödvändiga kunskaper för att utföra uppgiften)
och deadline för inlämning.'' Detta kommer också att betyda att vi
sannolikt behöver minska antalet uppgifter studenterna behöver göra. Min
uppfattning är att det är väldigt mycket uppgifter i de kurser som går
parallellt med mina kurser, vilket tillintetgör vad detta syftar till
att åstadkomma.

Men detta är nödvändigt enligt diskussionen om strategier för likvärdig
examination ovan.

\subsubsection{Varierad examination}\label{varierad-examination}

Vanligtvis är examinationsformerna varierade i en kurs. Det finns
normalt labbar (muntlig eller skriftlig redovisning), seminarier,
projekt (ofta med skriftlig och muntlig redovisning), tentamen. Dock
måste alla examinationsformer genomföras av alla studenter och med
godkänt resultat i samtliga. Detta tar lite udden av vad vi vill uppnå
med varierad examination.

Sett ur detta perspektiv bör vi ha examination som är varierad, men att
en student inte behöver få godkänt i varje examinationsmoment som rör
samma lärandemål. Detta skulle således motsvara en slags
``maximum''-policy för betygskriterierna (eller ``mastery-based
grading''). Det räcker med att ha nått A-nivån i betygskriterierna på
\emph{en} examinationsuppgift för ett lärandemål, man behöver inte nå A
på samtliga examinationsuppgifter för det lärandemålet. Rent krasst
skulle en student då bara behöva göra de uppgifter som examineras på det
sätt som passar dem bäst. De måste dock klara en uppgift per lärandemål,
då alla lärandemål måste examineras.

Notera att detta dock inte säger något om hur A på ett lärandemål och C
på ett annat ska sammanfattas i ett slutbetyg för hela kursen. Men de
två mest resonliga policyerna är ``minimum''- eller ``medel''-policyn.
``Minimum'' tar den lägsta betygsnivån uppnådd i alla lärandemål.
``Medel'' tar istället medel; dvs på några lärandemål uppnådde studenten
mer, på andra mindre. Om man tittar på användbarheten för det
sammanfattande betyget så är ``mimimum''-policyn mest relevant, då vet
man att studenten ligger på minst den nivån i alla lärandemål.
``Medel''-policyn tappar för mycket information.

Man bör definitivt reflektera över varför vi examinerar på ett visst
sätt och fundera över alternativ.

\subsubsection{Normkritiskt förhållningssätt}%
\label{normkritiskt-förhållningssätt}

I de flesta fall i min examination har nog kvinnliga studenter mest
fördelar. Men det är nog för att jag upplever dem som bättre; de kan
tänka utanför lådan, se olika perspektiv, förklara och berätta saker
bättre än de manliga studenterna.

Det jag tänker mest på i normkritisk mening i min egen praktik är att
inte skräddarsy utbildningarna efter barnlösa gymnasieungdomar, vilket
är normstudenten på våra utbildningar.
Och sannolikt är orsaken till att vår normstudent ser ut så hur vi har utformat 
utbildningarna.
Jag har sett andra utbildningar där normstudenten ser väldigt annorlunda ut.

\subsubsection{Fördjupad läsning om likvärdig examination}%
\label{fördjupad-läsning-om-likvärdig-examination}

Som nämndes ovan, fokuset på förhållandet mellan kvinnor och män
osynliggör icke-binära (och andra som inte följer normen):

``Data från KTH: Antal studerande kvinnor respektive män på KTH: KTH i
siffror''

Men som sagt, vi får inte mäta könsidentitet heller. Men då allt
data handlar om det så påverkar det ju fokus.

\subsection{Lokala regler för examination}%
\label{lokala-regler-examination}

Angående frågan i det andra quizzet:

Anta att det i examinationen i din kurs ingår en inlämningsuppgift och
att det i betygskriterierna står att om studenterna är sena med att
lämna in den så sänks deras betyg på den uppgiften, även om uppgiftens
kvalitet (det studenterna presterat) motsvarar ett högre betyg. Kan man
sänka betyget på inlämningsuppgiften enbart baserat på att den kommer in
efter deadline?

Även om examinator får göra det betyder det inte att examinator bör göra
det. Betygen ska reflektera lärandemålen, där betygskriterierna ställer
upp någon typ av progression relaterad till respektive lärandemål. Om
tiden är en viktig aspekt ska den vara en del av lärandemålen,
exempelvis att ett lärandemål handlar om projektledning (där
tidsaspekten är relevant). Man skulle också kunna argumentera för att
nivån av skicklighet kan relateras till hur snabbt en uppgift kan lösas.

I båda dessa fall ska i sådant fall tidsmätningen göras på ett korrekt
och gediget sätt. De individuella förutsättningarna ska också tas i
beaktande: det vore inte särskilt rättvisande om en student får ett
sämre betyg bara för att en närstående gått bort under kursens gång.
Dessa förutsättningar är dessutom såpass individuella så att det är
svårt att sätta upp några rättssäkra kriterier kring dem. Följaktligen
måste tidsmätningen anpassas.

En annan aspekt hittar vi i kap 6 högskoleförordningen:
\blockquote{%
  Antal prov m.m.

  21 § Om en högskola begränsar det antal tillfällen som en student får
  genomgå prov för att få godkänt resultat på en kurs eller del av en
  kurs, skall antalet tillfällen bestämmas till minst fem. Om godkänt
  resultat på en kurs eller del av en kurs förutsätter att studenten
  genomgått praktik eller motsvarande utbildning med godkänt resultat,
  skall antalet praktik- eller motsvarande utbildningsperioder bestämmas
  till minst två. Förordning (2006:1053).%
}

Det vill säga, en student har alltid rätt att göra ett nytt prov. Då
tidsaspekten är ett lärandemål som ska mätas och tidsmätningen bör göras
korrekt, så måste även tidsmätningen göras om från början. Tidsmätningen
kan inte fixeras till kursens början, \emph{utan till (om-)provets
början}.

Man kan ha viss tidsaspekt med i betygskriterierna även om det inte är
ett direkt lärandemål. Exempelvis att lösa ett problem och betyget
avgörs hur snabbt man löser problemet. Men då måste det vara en
kontrollerad och korrekt tidsmätning som inte påverkas av irrelevanta
faktorer (exempelvis planeringsförmåga eller skicklighet att hantera
oförutsägbara händelser i privatlivet).

Redan här finner jag det således absurt att tidsaspekten ska påverka
betyget i majoriteten av fall. Men vi hittar ytterligare intressanta
aspekter i UKÄ:s vägledning för rättssäker examination (sidan 34):

UKÄ konstaterar att högskoleförfattningarna visserligen inte reser några
formella hinder mot att tillämpa olika betygsskalor vid prov respektive
omprov. Men mot bakgrund av de starka invändningar som kan resas mot en
ordning med olika betygsskalor bör ett lärosäte noggrant överväga alla
aspekter och konsekvenser innan man inför en sådan ordning på en kurs.
Om lärosätet väljer att införa en ordning med olika betygsskalor på en
kurs, bör det framgå av kursplanen.

Några av argumenten från de positiva lärosätena är (sidan 34):

Det handlar främst om examinationsuppgifter där tidsaspekten är
avgörande för bedömningen av studentens prestationer. Argumentet är att
det inte ska vara möjligt att köpa sig mer tid vid t.ex. en
projektuppgift genom att inte gå upp på ordinarie examination

Vilket löses genom att göra en \emph{korrekt} tidsmätning.

Vi kan också påpeka att KTH är ett av de positiva lärosätena (sidan 34):

Även KTH och Högskolan i Jönköping anser att det i vissa speciella fall
bör vara möjligt med olika betygsskalor.

Om vi då återgår till en del av motiveringen i svaret till frågan, där
skrivs följande:

Det ska framgå av kursplanen eller av betygskriterierna vad
konsekvenserna blir om uppgiften inte lämnats in i tid. Konsekvensen kan
till exempel bli underkänt eller lägre betyg på uppgiften. (Riktlinje om
kursplan, betygssystem och examination inom alla utbildningsnivåer.)

Detta anser jag således vara väldigt förenklat och leder fel i det
generella fallet. Jag skulle säga att det är en begränsad tolkning som
inte tar hänsyn till studenternas rättigheter enligt
högskoleförordningen och är inte heller förenligt med målrelaterad
bedömning.
