\title{Motivering för kursupplägget}
\author{Daniel Bosk}
\institute{KTH EECS}

\mode*

\begin{frame}
  \maketitle
\end{frame}

\begin{abstract}
  % What's the problem?
% Why is it a problem? Research gap left by other approaches?
% Why is it important? Why care?
% What's the approach? How to solve the problem?
% What's the findings? How was it evaluated, what are the results, limitations, 
% what remains to be done?

% XXX Summary
\emph{Summary:}
\dots

% XXX Motivation and intended learning outcomes
\emph{Intended learning outcomes:}
\dots

% XXX Prerequisites
\emph{Prerequisites:}
\dots

% XXX Reading material
\emph{Reading:}
\dots

\end{abstract}

\begin{frame}
  \only<presentation>{\tableofcontents[hideallsubsections]}
  \only<article>{\tableofcontents*}
\end{frame}
\clearpage


\section{Introduktion}

Kursen har följande lärandemål (från kursplanen våren 2025):

Efter godkänd kurs ska studenten kunna
\begin{enumerate}
\item
  konstruera program utan kodupprepningar,
\item
  dela upp ett större problem i hanterliga delar,
\item
  dela upp ett program,
\item
  tillämpa styrstrukturer,
\item
  utforma och presentera användarvänliga utdata,
\item
  skapa flexibla applikationer,
\item
  konstruera programmerarvänliga program med lämpliga namn och
  kommentarer,
\item
  konsekvent språk och typografi,
\item
  konstruera interaktiva program,
\item
  använda och konstruera sammansatta datatyper och klasser,
\item
  överföra data mellan fil och program,
\item
  granska andras program,
\end{enumerate}
i syfte att senare
\begin{itemize}
\item
  kunna använda programmering för att lösa problem,
\item
  kunna tillämpa problemlösningsmetodiken även inom andra områden än
  programmering,
\item
  kunna diskutera programutveckling med experter,
\item
  kunna bedöma program i storleksordningen på ca femhundra rader kod,
\item
  självständigt och i grupp kunna lösa problem genom att konstruera
  program på upp till femhundra rader i ett modernt programspråk.
\end{itemize}
 
Denna diskussion kommer att utgå från några av mina reflektioner från 
examinatorskursen, återfinns i \cref{examinatorskursen}.


\section{Utformningen av datorprovet\protect\footnote{%
    Följande examinatorer har bidragit till datorprovets nuvarande utformning 
    (bokstavsordning på efternamn):
    Daniel Bosk, Olof Bälter, Vahid Mosavat, Emma Riese.
    Daniel Bosk och Vahid Mosavat lade till att studenterna får skriva 
    kvarvarande lärandemål.
    Emma Enström har bidragit till diskussionen om det.
}}

Ladok-momentet LAB2, 1.5 hp, är ett datorprov.
Studenterna får skriva provet på dator.
Frågorna berör kursens olika lärandemål (dock inte alla) och kompletterar 
laborationer och projekt.

\subsection{Målrelaterad bedömning och validitet}

Provet bedömer lärandemålen enligt \cref{LärandemålDatorprovet}.
Provet har 21 frågor totalt.
Det finns tre frågor per lärandemål.
För jämförelse har vi kursens lärandemål i \cref{LarandemalKursen}.

\begin{table}[p]
  \caption{Mappning mellan lärandemål och frågor för datorprovet.}
  \label{LärandemålDatorprovet}
  \begin{tabular}{rll}
    \toprule
    \(n\) & Lärandemål & Frågor\\
    \midrule
    1 & Programmerarvänliga program och tilldelningar & Frågorna 1-3\\
    2 & Tillämpa styrstrukturer & Frågorna 4-6\\
    3 & Utforma och presentera användarvänliga utdata & Frågorna 7-9\\
    4 & Överföra data mellan fil och program & Frågorna 10-12\\
    5 & Konstruera interaktiva program & Frågorna 13-15\\
    6 & Använda och konstruera sammansatta datatyper och klasser & Frågorna 
    16-18\\
    7 & Dela upp ett program & Frågorna 19-21\\
    \bottomrule
  \end{tabular}
\end{table}

\begin{table}[p]
  \caption{Lärandemål för kursen.}
  \label{LarandemalKursen}
  \begin{tabular}{rp{0.9\columnwidth}}
    \toprule
    \(n\) & Lärandemål\\
    \midrule
    1 & Konstruera program utan kodupprepningar\\
    2 & Dela upp ett större problem i hanterliga delar\\
    3 & Dela upp ett program\\
    4 & Tillämpa styrstrukturer\\
    5 & Utforma och presentera användarvänliga utdata\\
    6 & Skapa flexibla applikationer\\
    7 & Konstruera programmerarvänliga program med lämpliga namn och kommentarer\\
    8 & Konsekvent språk och typografi\\
    9 & Konstruera interaktiva program\\
    10 & Använda och konstruera sammansatta datatyper och klasser\\
    11 & Överföra data mellan fil och program\\
    12 & Granska andras program\\
    \bottomrule
  \end{tabular}
\end{table}

Om vi jämför \cref{LärandemålDatorprovet} med \cref{LarandemalKursen} ser vi 
att datorprovet inte täcker följande lärandemål:
\begin{enumerate}
\setcounter{enumi}{0}
\item Konstruera program utan kodupprepningar \label{item:reuse}
\item Dela upp ett större problem i hanterliga delar \label{item:divide}
\setcounter{enumi}{5}
\item Skapa flexibla applikationer \label{item:flexible}
\item Konstruera programmerarvänliga program med lämpliga namn och kommentarer \label{item:readable}
\item Konsekvent språk och typografi \label{item:consistency}
\setcounter{enumi}{11}
\item Granska andras program \label{item:review}
\end{enumerate}
Dessa måste således täckas av andra examinationsmoment.

Studenterna kan få ett bonuspoäng per lärandemål (i 
\cref{LärandemålDatorprovet}).
Uppgifterna som de måste göra (i tid) för att få bonuspoängen är relaterade 
till just det lärandemålet.

För att klara provet behöver studenterna klara minst 2 av 3 frågor per 
lärandemål.
Med bonuspoängen behöver de då bara klara 1 av 3 frågor per lärandemål.

\paragraph{Analys}

Utifrån diskussionen i \cref{målrelaterad-bedömning-validitet} kan vi 
konstatera att datorprovet är utformat på ett sätt som ger hög validitet.
Det går inte att klara datorprovet utan att ha uppnått godkänt resultat på 
samtliga lärandemål.

Bonuspoängens utformning går också i linje med validiteten.
Bonuspoängen är per lärandemål och kan således inte bidra till att studenten 
får godkänt resultat om de inte uppnått godkänt resultat på samtliga 
lärandemål.
Som diskuteras nedan, \cref{BehövsDatorprovet}, kan vi dock diskutera huruvida 
bonuspoängen sänker provets validitet:
då övervakad examination blir oövervakad.

Enström påtalade att det är skillnad mellan att klara enstaka frågor jämfört 
med ett helt prov.
Exempelvis är det lättare att gissa sig till godkänt på ett lärandemål (gissa 
rätt på två av tre frågor) åt gången än alla frågor samtidigt (gissa rätt på 14 
av 21 frågor).
Eller med bonuspoäng:
gissa rätt på en fråga åt gången istället för sju vid samma tillfälle.

Vi kan dock notera att inga lärandemål har tidsaspekter.
Tiden det tar för studenten att genomföra en uppgift för ett lärandemål är 
därför inte relevant för bedömningen.
Provet är dock tidsbegränsat, men det ska ses som administrativ konsekvens.
Således är det inget problem att studenterna får olika mycket tid på sig att 
lösa uppgifterna.
Vi kan således låta studenterna skriva om endast de lärandemål som de inte ännu 
uppnått godkänt på, då lärandemålen examineras oberoende av varandra (skilda 
frågor).
Men som Enström påpekade påverkar det provet validitet i form av möjlighet att 
gissa sig till ett godkänt resultat.


\subsection{Rättssäker examination}

Det är flera saker vi bör diskutera här.
Examinatorskursen tar upp proportionalitetsprincipen och 
likabehandlingsprincipen.
Vi har även UKÄ:s skrift \emph{Rättssäker examination} att luta oss på.

\subsubsection{Proportionalitetsprincipen}

Proportionalitetsprincipen innebär att examinationen ska stå i proportion till 
vad den mäter.
Som vi såg i \cref{proportionalitetsprincipen} ovan, kan vi ställa oss följande 
frågor:
\begin{itemize}
  \item Behövs åtgärden för att myndigheten ska uppnå de mål som gäller för 
    den?
  \item Finns det mindre ingripande alternativ som ger samma eller nästan samma
    resultat?
  \item Är skadorna som uppstår rimliga i förhållande till det mål som uppnås?
\end{itemize}

\paragraph{Behövs datorprovet?}\label{BehövsDatorprovet}

Om provet ges oövervakat tillför det inte särskilt mycket, då samtliga 
lärandemål redan bedöms i, främst, p-uppgiften.
Övervakat prov tillför dock att vi bedömer dem övervakat.
Laborationerna och p-uppgiften kan de lösa med, möjligen otillåtna, hjälpmedel.
Att de klarar provet övervakat visar att de kan lösa uppgifterna på egen hand.
På så vis kan vi motivera att provet behövs och således att den första frågan 
besvaras med ja.

Dock går dessa argument emot användandet av bonuspoäng, då en del av provet går 
från övervakat till oövervakat.

\paragraph{Finns andra alternativ?}

Huvudpoängen är att vi vill bedöma att de kan lösa uppgifterna på egen hand, 
utan otillåtna hjälpmedel.
Då måste vi övervaka dem när de löser någon sorts uppgift.
Vi måste också bedöma flertalet lärandemål för att få en rättvisande bild av 
deras kunskaper.
Provets utformning gör det också enkelt att bedöma, vilket är positivt ur 
resursanvändningssynpunkt.

\paragraph{Hur kan vi hantera skador?}

Hur är det då med eventuella skador som uppstår?
Det är ett prov, som bara kan skrivas vid vissa tillfällen (för övervakningen).
Provet håller inne på 1.5 hp för studenten.
Vi vet från erfarenhet att ibland kan så lite som 0.5 hp vara avgörande för 
huruvida studenten får CSN eller inte.
Denna typ av examinationsmoment kan således orsaka en hel del stress för 
studenten.
Det kan även förekomma personliga angelägenheter som förhindrar att studenten 
kan skriva på något av de (få) tillfällena som finns.
Vi återkommer till detta när vi diskuterar likhetsprincipen och normkritiskt 
förhållningssätt, men exempelvis småbarnsföräldrar kan få förhinder pga vab.

Vi kan reducera de skador som datorprovet för med sig genom att öka antalet 
tillfällen som studenten kan skriva provet.
Då blir det en kostnadsfråga för myndigheten.
Jag anser att vi inte behöver ha fler än tre tillfällen per kurstillfälle per 
år om vi låter samtliga studenter skriva på samtliga provtillfällen.
När vi ändå organiserar övervakning för att ge provet, då kan alla som behöver 
få skriva det.
Den kostnadsökningen blir då marginell för myndigheten och kan inte användas 
för att motivera att neka studenter att delta vid provtillfällen.

Det ges för tillfället fem kurstillfällen per år.
Vi kan låta varje kurstillfälle ha två tillfällen för datorprovet.
Det skulle ge totalt tio tillfällen per år,
samt ett extra i augusti för samtliga.
Då skulle vi kunna slå ihop några av de tillfällen som ges under höstterminen, 
när flertalet kurstillfällen ges, för att få en jämnare spridning över året.
Vi vill att det ska ha gått iallafall tre veckor mellan 
provtillfällena\footnote{%
  Det finns inga nationella regler men UKÄ anser att tiden mellan meddelandet 
  av tentamensresultatet och omprovet måste vara minst tio arbetsdagar eller 
  minst två veckor.
}.

\subsubsection{Likhetsprincipen}

Enligt likhetsprincipen (i offentlig förvaltning) ska alla studenter behandlas 
lika, som i \emph{allas likhet inför lagen}.
Det vill säga alla studenter måste klara samma krav för att få godkänt, dvs 
måste uppnå godkänt på alla lärandemål.
Detta förutsätter dock inte att alla behöver genomgå samma examination, vilket 
för oss in på nästa ämne.

\subsection{Likvärdig examination}

Det finns några aspekter att ta upp här:
\begin{itemize}
  \item inkluderande examination,
  \item varierad examination,
  \item normkritiskt förhållningssätt.
\end{itemize}

\subsubsection{Normstudenten och inkluderande examination}

Låt oss se hur studenterna ser ut på våra program.
Vi ser antalet sökande och antalet antagna i olika åldersgrupper i 
\cref{ÅlderStudenter}.
Vi kontrasterar studenterna i prgi och prgm med studenterna på 
kandidatprogrammet i Systemvetenskap vid LTU\footnote{%
  Programmet är det med flest, i antal, sökande över 34 år inför höstterminen 
  2024, nationellt bland program med datainriktning.
}.
Detta program har studieorten \enquote{Ortsoberoende} och ges alltså på 
distans.

\begin{table}
  \caption{%
    Antal sökande och antagna per program samt åldersgrupp.
    Statistiken är hämtad från UHR:s antagningsstatistik för hösten 2024.%
  }%
  \label{ÅlderStudenter}
  \centering
  \begin{tabular}{lrrr}
    \toprule
    & \textbf{Under 25} & \textbf{25--34} & \textbf{Över 34} \\
    \midrule
    \textbf{prgi Sökande} & 2873 & 151 & 26 \\
    \textbf{prgi Antagna} & 201 & 8 & 1 \\
    \midrule
    \textbf{prgm Sökande} & 1906 & 126 & 27 \\
    \textbf{prgm Antagna} & 184 & 4 & 2 \\
    \midrule
    \textbf{LTU Sökande} & 448 & 928 & 536 \\
    \textbf{LTU Antagna} & 9 & 24 & 12 \\
    \bottomrule
  \end{tabular}
\end{table}

Vi kan konstatera att det finns fog att fråga oss huruvida våra val vid design 
av undervisning och examination påverkar vad som är normstudenten på våra 
program.
Och ett normkritiskt förhållningssätt innebär att vi inte borde anpassa oss 
till majoriteten, utan ta dem som avviker från normen i beaktande.

\paragraph{Antalet prov}

Jag skulle därför argumentera att vi bör ge studenterna fler än två tillfällen 
per år för datorprovet.
Just för att de få som är äldre, med stor sannolikhet, har mer komplexa 
livssituationer än de yngre studenterna:
barn som måste hämtas, lämnas, vabbas, jobb på sidan om för att klara 
bolånet\footnote{%
  Även om CSN försöker att avhjälpa dessa situationer med tilläggslån, så är 
  jag inte helt övertygad att banken godtar det när man har bolån.
} etc.
Således är det lättare hänt för dessa studenter att missa ett provtillfälle.

Det kan också argumenteras att även de yngre har ett incitament att skriva 
proven sjuka, om de är sjuka vid ett av tillfällena, när det ges så få 
tillfällen.

Ett flertal provtillfällen per år, som är kända i förväg, underlättar för alla 
studenter att planera och hantera oförutsedda händelser.

Bosk, Riese och Mosavat kom fram till att vi ger provet i varje period (i 
tentaperioden) samt i augusti.
Då får samtliga studenter fem skrivtillfällen per år.

\paragraph{Bonuspoäng}

Jag argumenterar fortfarande att bonuspoängen ger fördel för de 
\enquote{normala} studenterna, som lättare kan hålla strikta deadlines och 
lättare kan följa kursens schema utan avvikelser.
Texten om inkluderande examination använder som exempel just att vi bör vara 
mer generösa med tid för att, exempelvis föräldrar, lättare ska kunna anpassa 
studierna till de problem som uppstår i vardagen.
Bonuspoäng med strikt deadline anser jag går emot det.

Det är just deadline som jag är emot.
Som jag skrev ovan kan jag acceptera att en del av examinationen flyttas från 
det övervakade provet till det oövervakade studiematerialet.
Jag skulle således säga att jag förespråkar att ge bonuspoäng till alla som 
gjort uppgifterna i studiematerialet innan provet.


\subsubsection{Varierad examination}

Rieses design med tre frågor per lärandemål tar detta i beaktande.
De tre frågorna är av olika typ, men alltid samma tre typer.
Detta för att studenter med olika svårigheter för olika typer av frågor ska 
kunna visa sina kunskaper.
Därav gränsen på två av tre frågor per lärandemål för att nå godkänt.

I de fall där studenter behöver andra format, exempelvis muntlig istället för 
skriftlig examination, så har vi primärt Funka.
Det finns dock även möjlighet för de studenter som har svårt att klara ett 
skriftligt prov under tidspress att få alternativ examination.
I kursplanen står det att \enquote{\textins{e}xaminator får medge annan 
examinationsform vid omexamination av enstaka studenter.}

\begin{frame}[allowframebreaks]
  \printbibliography[heading=bibintoc]
\end{frame}

\appendix

\section{Utvalda reflektioner från examinatorskursen}%
\label{examinatorskursen}

Detta är delar av mina reflektioner från examinatorskursen.

\subsection{Målrelaterad bedömning och validitet}%
\label{målrelaterad-bedömning-validitet}

\paragraph{Validitet i examination}

Det exempel som ges är bra, men det skulle
behöva kompletteras av ytterligare ett exempel. Det behövs ett exempel
på låg validitet (för att ge den nödvändiga kontrasten enligt
variationsteorin). Majoriteten av tentor som ges på KTH har med stor
sannolikhet låg validitet, då många är utformade på följande sätt. Vi
ger basfallet med två lärandemål och resten följer av induktion.

Lärandemål:
\begin{itemize}
\item
  Efter godkänd kurs ska studenten kunna redogöra för \(X\).
\item
  Efter godkänd kurs ska studenten kunna analysera \(Y\).
\end{itemize}

Examination:
\begin{itemize}
\item
  Tio frågor på ett prov, utformade så att måluppfyllelse kan visas.
\item
  Fem täcker \(X\) och fem täcker \(Y\).
\end{itemize}

Operationellt bedömningskriterium:
\begin{itemize}
\item
  Minst 6 av dessa 10 frågor ska vara korrekt besvarade för godkänt.
\end{itemize}

Denna utformning ger låg validitet då en student kan besvara fem frågor
om \(X\) korrekt, men bara en fråga om \(Y\); och trots detta får godkänt. 
Genom induktion får vi att detsamma gäller för N lärandemål. Således ger
tentor som slentrianmässigt har 50\% av poängen för E låg validitet.

\paragraph{Bonuspoäng}

Låt oss även utforska tillämpningen av bonuspoäng. Om bonuspoängen ges
som extra poäng på provet, godtyckligt, utan att motsvara en specifik
fråga (dvs ett specifikt lärandemål), då förstör även detta validiteten
i examinationen, på samma sätt som ovan. För att bonuspoängen inte ska
degradera validiteten måste varje bonuspoäng motsvara en specifik fråga
som motsvarar ett specifikt lärandemål. Uppgiften som ger bonus måste
också examinera just det lärandemålet. Dvs ett korrekt utformat
bonussystem flyttar (delar av) examinationen av lärandemål från ett
tillfälle till ett annat.

\subsection{Rättssäker examination}\label{rättssäker-examination}

\subsubsection{Proportionalitetsprincipen}\label{proportionalitetsprincipen}

Kan vi motivera användningen av bonuspoäng?

\begin{itemize}
\item
  Behövs åtgärden för att myndigheten ska uppnå de mål som gäller för
  den?

  \begin{itemize}
  \item
    Nej, vi kan examinera ändå.
  \item
    Ja, för vi vill ha genomströmning.
  \end{itemize}
\item
  Finns det mindre ingripande alternativ som ger samma eller nästan
  samma resultat?

  \begin{itemize}
  \item
    Ja, vi kan lära studenterna att planera sin tid, vilket dessutom
    borde vara mer värdefullt.
  \end{itemize}
\item
  Är skadorna som uppstår rimliga i förhållande till det mål som uppnås?

  \begin{itemize}
  \item
    Nej, det orsakar onödig stress och förstör möjligheten att planera
    för studenterna när vi har överdrivet mycket examination och det är
    examination hela tiden som ska vara klar just då för att få
    bonuspoäng. (Istället för att vi låter dem ansvara själv för sitt
    lärande.)
  \end{itemize}
\end{itemize}

Vi kan även fråga oss om vi behöver all den redundanta examination som
vi har i programmeringskurserna (labbar, datorprov och p-uppgift) och
andra kurser.

Vi återkommer till detta.

\subsubsection{Likhetsprincipen}\label{likhetsprincipen}

Är det motiverat att tillämpa bonuspoäng sett ur likhetsprincipens
perspektiv? (Denna diskussion är även högst relevant för området
Likvärdig examination, \cref{likvärdig-examination}, det området borde vara en 
konsekvens av denna princip.) Det är lättare för en barnlös 20-åring att hinna 
med alla deadlines skapade av bonuspoäng jämfört med en ensamstående
småbarnsmamma.

Rent teoretiskt borde inte bonuspoängen i sig inte ge några fördelar ur
examinationens perspektiv (\cref{målrelaterad-bedömning-validitet}). Detta är 
dock inte sant, men vi återkommer
till det. Låt oss anta att det är sant, då borde den ensamstående
småbarnsmamman kunna få samma betyg som den barnlösa 20-åringen utan
skillnad i prestation. Men då kan vi fråga varför vi har bonuspoängen
överhuvudtaget?

Om man tillämpar bonuspoängen på ett korrekt sätt så innebär de bara att
man flyttar examination (eller delar av) från ett tillfälle till ett
annat: I fallet med ett prov (eller tenta, som är vanligast att använda
bonuspoäng till) så flyttar man examinationen av en fråga (lärandemål)
från provet till tillfället då studenten kunde tillgodogöra sig
bonuspoängen.

Till och med här, i det optimala fallet, har den ensamstående
småbarnsmamman missgynnats med \emph{mindre tid per fråga} på provet,
jämfört med den barnlösa 20-åringen. Detta då tiden för provet är
konstant i praktiken, medan antalet frågor varierar på grund av
bonuspoängen. Vi kan även ta upp påverkan av andra faktorer som stress,
även om de är svårare att kvantifiera.

Låt oss nu titta på en mindre optimal tillämpning av bonuspoäng som är
mer lik hur bonuspoäng tillämpas i praktiken i många kurser.

Bonuspoängen ges som extra poäng på provet, godtyckligt, utan att
motsvara specifika lärandemål. Som sagt ovan 
(\cref{målrelaterad-bedömning-validitet}), förstör detta validiteten i 
examinationen.
Detta får ytterligare effekter sett ur likhetsperspektiv, då
bonussystemet nu inte längre flyttar examinationen från ett tillfälle
till ett annat. Att hinna i tid kan ge fördelar i helt andra delar av
examinationen. Det kan dessutom göra så att den barnlösa 20-åringen kan
få poäng för samma kunskaper/färdigheter två gånger (om studenten gör
frågan som motsvarar bonuspoängen igen på tentan utan att detta dras
av), vilket kan påverka betyget till det högre.

Sedan kan vi ju återvända till proportionalitetsprincipen för att
diskutera om fördelarna för de barnlösa 20-åringarna väger över
nackdelarna för de ensamstående småbarnsföräldrarna. Med tanke på att
den sociala snedrekryteringen till högskolan är ett genomgående problem,
skulle jag säga att det inte är värt det.

\subsection{Likvärdig examination}\label{likvärdig-examination}

\paragraph{Diskriminering}

Det som jag finner intressant är att fokus här väldigt
lätt låses till de olika diskrimineringsgrunderna. I min värld är
diskrimineringsgrunderna lågvattenmärket vi behöver hålla, helst bör vi
göra bättre än bara de sex diskrimineringsgrunderna. Exempelvis, den
sociala snedrekryteringen är inte diskriminering, men vi bör ändå göra
något åt det. Det är lite som med fokuset på mansdominansen inom
akademin/teknikområdet/etc., vilket leder fokuset till män och kvinnor,
men helt glömmer bort icke-binära och könsidentitet generellt. (En
anledning är väl att det inte går att mäta, då vi inte får fråga om det
datat; medan personnumret kan användas för att läsa ut man eller
kvinna.)

Diskrimineringsgrunderna måste man ta hänsyn till för att få ett E. Men
vi bör ha ambitionen att hamna på A.

\subsubsection{Strategier för likvärdig examination}%
\label{strategier-likvärdig-examination}

En väldigt bra sida, men \ldots{}

``Vi kan \emph{lära mer om} olika studentgrupper för att bättre lära
känna våra studenter och utforma examinationen inkluderande, vi kan
anpassa examinationen för olika studentgrupper'', men det blir svårt att
lära känna de studenter som är exkluderade på grund av
utbildningens/undervisningens generella utformande (campus kl 8--17 med
obligatorisk närvaro).

Policyn att deadline alltid ska vara kl 19 på vardagar är ett intressant
steg. Dock så vet jag inte om det hjälper småbarnsföräldrar så som det
är avsett. I min erfarenhet är det flexibiliteten som underlättar för
dem, rigida strukturer med tighta deadlines är svåra oavsett när på
dygnet de är.

En sak vi behöver göra på detta område är att se över hur mycket tid
våra kurser faktiskt tar i anspråk. Finns det faktiskt tid över till att
göra uppgifter utanför schemalagd tid? Vissa program är i vissa perioder
schemalagda kl 8--17, varje dag---då finns inte tid att göra något på
kursen när man inte är schemalagd.

I mitt eget fall schemalade jag egen studietid för att skydda
studenterna från andra kurser. Visade sig att de ändå bara hade tid för
två timmar utanför kursens ordinarie schemalagda tillfällen (en
föreläsning, en labb och en övning).

\subsubsection{Inkluderande examination}\label{inkluderande-examination}

``Ett annat exempel på inkluderande examination är att ge gott om tid
för inlämningsuppgifter, alltså från det att uppgiften publiceras (givet
att studenterna då har nödvändiga kunskaper för att utföra uppgiften)
och deadline för inlämning.'' Detta kommer också att betyda att vi
sannolikt behöver minska antalet uppgifter studenterna behöver göra. Min
uppfattning är att det är väldigt mycket uppgifter i de kurser som går
parallellt med mina kurser, vilket tillintetgör vad detta syftar till
att åstadkomma.

Men detta är nödvändigt enligt diskussionen om strategier för likvärdig
examination ovan.

\subsubsection{Varierad examination}\label{varierad-examination}

Vanligtvis är examinationsformerna varierade i en kurs. Det finns
normalt labbar (muntlig eller skriftlig redovisning), seminarier,
projekt (ofta med skriftlig och muntlig redovisning), tentamen. Dock
måste alla examinationsformer genomföras av alla studenter och med
godkänt resultat i samtliga. Detta tar lite udden av vad vi vill uppnå
med varierad examination.

Sett ur detta perspektiv bör vi ha examination som är varierad, men att
en student inte behöver få godkänt i varje examinationsmoment som rör
samma lärandemål. Detta skulle således motsvara en slags
``maximum''-policy för betygskriterierna (eller ``mastery-based
grading''). Det räcker med att ha nått A-nivån i betygskriterierna på
\emph{en} examinationsuppgift för ett lärandemål, man behöver inte nå A
på samtliga examinationsuppgifter för det lärandemålet. Rent krasst
skulle en student då bara behöva göra de uppgifter som examineras på det
sätt som passar dem bäst. De måste dock klara en uppgift per lärandemål,
då alla lärandemål måste examineras.

Notera att detta dock inte säger något om hur A på ett lärandemål och C
på ett annat ska sammanfattas i ett slutbetyg för hela kursen. Men de
två mest resonliga policyerna är ``minimum''- eller ``medel''-policyn.
``Minimum'' tar den lägsta betygsnivån uppnådd i alla lärandemål.
``Medel'' tar istället medel; dvs på några lärandemål uppnådde studenten
mer, på andra mindre. Om man tittar på användbarheten för det
sammanfattande betyget så är ``mimimum''-policyn mest relevant, då vet
man att studenten ligger på minst den nivån i alla lärandemål.
``Medel''-policyn tappar för mycket information.

Man bör definitivt reflektera över varför vi examinerar på ett visst
sätt och fundera över alternativ.

\subsubsection{Normkritiskt förhållningssätt}%
\label{normkritiskt-förhållningssätt}

I de flesta fall i min examination har nog kvinnliga studenter mest
fördelar. Men det är nog för att jag upplever dem som bättre; de kan
tänka utanför lådan, se olika perspektiv, förklara och berätta saker
bättre än de manliga studenterna.

Det jag tänker mest på i normkritisk mening i min egen praktik är att
inte skräddarsy utbildningarna efter barnlösa gymnasieungdomar, vilket
är normstudenten på våra utbildningar.
Och sannolikt är orsaken till att vår normstudent ser ut så hur vi har utformat 
utbildningarna.
Jag har sett andra utbildningar där normstudenten ser väldigt annorlunda ut.

\subsubsection{Fördjupad läsning om likvärdig examination}%
\label{fördjupad-läsning-om-likvärdig-examination}

Som nämndes ovan, fokuset på förhållandet mellan kvinnor och män
osynliggör icke-binära (och andra som inte följer normen):

``Data från KTH: Antal studerande kvinnor respektive män på KTH: KTH i
siffror''

Men som sagt, vi får inte mäta könsidentitet heller. Men då allt
data handlar om det så påverkar det ju fokus.

\subsection{Lokala regler för examination}%
\label{lokala-regler-examination}

Angående frågan i det andra quizzet:

Anta att det i examinationen i din kurs ingår en inlämningsuppgift och
att det i betygskriterierna står att om studenterna är sena med att
lämna in den så sänks deras betyg på den uppgiften, även om uppgiftens
kvalitet (det studenterna presterat) motsvarar ett högre betyg. Kan man
sänka betyget på inlämningsuppgiften enbart baserat på att den kommer in
efter deadline?

Även om examinator får göra det betyder det inte att examinator bör göra
det. Betygen ska reflektera lärandemålen, där betygskriterierna ställer
upp någon typ av progression relaterad till respektive lärandemål. Om
tiden är en viktig aspekt ska den vara en del av lärandemålen,
exempelvis att ett lärandemål handlar om projektledning (där
tidsaspekten är relevant). Man skulle också kunna argumentera för att
nivån av skicklighet kan relateras till hur snabbt en uppgift kan lösas.

I båda dessa fall ska i sådant fall tidsmätningen göras på ett korrekt
och gediget sätt. De individuella förutsättningarna ska också tas i
beaktande: det vore inte särskilt rättvisande om en student får ett
sämre betyg bara för att en närstående gått bort under kursens gång.
Dessa förutsättningar är dessutom såpass individuella så att det är
svårt att sätta upp några rättssäkra kriterier kring dem. Följaktligen
måste tidsmätningen anpassas.

En annan aspekt hittar vi i kap 6 högskoleförordningen:
\blockquote{%
  Antal prov m.m.

  21 § Om en högskola begränsar det antal tillfällen som en student får
  genomgå prov för att få godkänt resultat på en kurs eller del av en
  kurs, skall antalet tillfällen bestämmas till minst fem. Om godkänt
  resultat på en kurs eller del av en kurs förutsätter att studenten
  genomgått praktik eller motsvarande utbildning med godkänt resultat,
  skall antalet praktik- eller motsvarande utbildningsperioder bestämmas
  till minst två. Förordning (2006:1053).%
}

Det vill säga, en student har alltid rätt att göra ett nytt prov. Då
tidsaspekten är ett lärandemål som ska mätas och tidsmätningen bör göras
korrekt, så måste även tidsmätningen göras om från början. Tidsmätningen
kan inte fixeras till kursens början, \emph{utan till (om-)provets
början}.

Man kan ha viss tidsaspekt med i betygskriterierna även om det inte är
ett direkt lärandemål. Exempelvis att lösa ett problem och betyget
avgörs hur snabbt man löser problemet. Men då måste det vara en
kontrollerad och korrekt tidsmätning som inte påverkas av irrelevanta
faktorer (exempelvis planeringsförmåga eller skicklighet att hantera
oförutsägbara händelser i privatlivet).

Redan här finner jag det således absurt att tidsaspekten ska påverka
betyget i majoriteten av fall. Men vi hittar ytterligare intressanta
aspekter i UKÄ:s vägledning för rättssäker examination (sidan 34):

UKÄ konstaterar att högskoleförfattningarna visserligen inte reser några
formella hinder mot att tillämpa olika betygsskalor vid prov respektive
omprov. Men mot bakgrund av de starka invändningar som kan resas mot en
ordning med olika betygsskalor bör ett lärosäte noggrant överväga alla
aspekter och konsekvenser innan man inför en sådan ordning på en kurs.
Om lärosätet väljer att införa en ordning med olika betygsskalor på en
kurs, bör det framgå av kursplanen.

Några av argumenten från de positiva lärosätena är (sidan 34):

Det handlar främst om examinationsuppgifter där tidsaspekten är
avgörande för bedömningen av studentens prestationer. Argumentet är att
det inte ska vara möjligt att köpa sig mer tid vid t.ex. en
projektuppgift genom att inte gå upp på ordinarie examination

Vilket löses genom att göra en \emph{korrekt} tidsmätning.

Vi kan också påpeka att KTH är ett av de positiva lärosätena (sidan 34):

Även KTH och Högskolan i Jönköping anser att det i vissa speciella fall
bör vara möjligt med olika betygsskalor.

Om vi då återgår till en del av motiveringen i svaret till frågan, där
skrivs följande:

Det ska framgå av kursplanen eller av betygskriterierna vad
konsekvenserna blir om uppgiften inte lämnats in i tid. Konsekvensen kan
till exempel bli underkänt eller lägre betyg på uppgiften. (Riktlinje om
kursplan, betygssystem och examination inom alla utbildningsnivåer.)

Detta anser jag således vara väldigt förenklat och leder fel i det
generella fallet. Jag skulle säga att det är en begränsad tolkning som
inte tar hänsyn till studenternas rättigheter enligt
högskoleförordningen och är inte heller förenligt med målrelaterad
bedömning.


