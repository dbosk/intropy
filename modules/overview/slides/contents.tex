\title{%
  Intro till en kurs i programmeringsteknik
}
\author{Daniel Bosk}
\institute{%
  KTH EECS
}

\mode<article>{\maketitle}
\mode<presentation>{%
  \begin{frame}
    \maketitle
  \end{frame}
}

\mode*

\begin{abstract}
  According to variation theory (see~\cite{NCOL}\footfullcite{NCOL}), learning an 
educational objective requires the learner to distinguish all aspects of that 
educational objective. What these aspects are is hard to tell for someone who 
has already mastered the educational objective in question. However, 
misconceptions occur when the learner cannot yet distinguish one or more 
critical aspects. Thus misconceptions can help us identify what those 
(critical) aspects are. Then we can teach the learner to distinguish those 
critical aspects by varying examples through a series of patterns of variation.

In our work (in progress), we explore the existing literature on misconceptions 
in introductory programming courses and analyse it through the lens of 
variation theory to identify the necessary aspects of programming that a 
learner must learn to distinguish. We also outline patterns of variation to 
teach to distinguish these aspects.

\Textcite{NCOL} also connects the patterns of variation of variation theory to 
both deep learning and scientific discoveries. In both cases, the learners (the 
researcher is also a learner of the unknown) introduce variation for themselves 
through these patterns of variation. We hypothesize the connection between the 
patterns of variation and the skill of debugging (the programmer is learning 
about some unknown when debugging).

\end{abstract}


\section{Vilka är vi?}

\subsection{Lärare}

\begin{frame}
  \begin{block}{Kursansvarig lärare}
    \begin{itemize}
      \item Namn: Daniel Bosk
      \item Lärarutbildad i matematik och datateknik (CL, 2006--2011) vid KTH 
        och Stockholms universitet (f.d. Lärarhögskolan)
      \item Forskarutbildad i datalogi (2014--2020) vid KTH
      \item Undervisat på universitetsnivå sedan 2011.
      \item Fokus på e-lärande och distansundervisning.
    \end{itemize}
  \end{block}
\end{frame}

\begin{frame}
  \begin{block}{Fler lärare på kursen}
    \begin{itemize}
      \item Olle
      \item Övningsledare
      \item Labbhandledare
    \end{itemize}
  \end{block}
\end{frame}

%\subsection{Studenter}
%
%\begin{frame}
%  \begin{exercise}[Vad har ni för intressen?]
%    \begin{itemize}
%      \item \url{https://menti.com}
%    \end{itemize}
%  \end{exercise}
%\end{frame}
%
%\begin{frame}
%  \begin{exercise}[Vilka är ni?]
%    \begin{itemize}
%      \item Break-out rooms: tre personer i varje.
%      \item Vart kommer ni från?
%      \item Varför kom ni hit?
%      \item Vart är ni på väg?
%    \end{itemize}
%  \end{exercise}
%\end{frame}


%\section{Varför ska ni läsa programmering?}
%
%\begin{frame}
%  \begin{center}
%    Varför ska ni lära er att programmera?
%  \end{center}
%\end{frame}


\section{Om kursen}

\subsection{Vad ska ni få ut?}

\begin{frame}
  \begin{block}<+>{Färdigheter}
    \begin{itemize}
      \item Algoritmiskt tänkande
      \item Problemlösningsförmåga
    \end{itemize}
  \end{block}

  \begin{block}<+>{Kunskaper}
    \begin{itemize}
      \item Terminologi
      \item Mental modell av datorsystem
    \end{itemize}
  \end{block}

  \begin{block}<+>{Verktyg}
    \begin{itemize}
      \item Python
      %\item Matlab
    \end{itemize}
  \end{block}
\end{frame}

\subsection{Hur ska ni få ut detta?}

\begin{frame}
  \begin{block}{Moduler}
    \begin{enumerate}
      \item Förberedelse
      \item Övning
      \item Laboration
      \item Fördjupande övning
    \end{enumerate}
  \end{block}

  \pause

  \begin{block}{Projekt}
    \begin{itemize}
      \item Större uppgift.
      \item Andra halvan av kursen.
    \end{itemize}
  \end{block}
\end{frame}

\begin{frame}
  \begin{block}{Förberedelse}
    \begin{itemize}
      \item Egna studier: OLI och FeedbackFruits.
      \item Interaktionen bra för lärande.
    \end{itemize}
  \end{block}

  \pause

  \begin{block}{Övningar}
    \begin{itemize}
      \item Genomgång av problemen som uppdagades under förberedelsen.
      \item Får information från interaktionen under era förberedelser.
      \item Fokuserar på de viktigaste delarna.
      \item Interaktiva tillfällen för lärande.
      \item Praktisk problemlösning för er!
    \end{itemize}
  \end{block}
\end{frame}

\begin{frame}
  \begin{block}{Laborationer}
    \begin{itemize}
      \item Ger återkoppling på ert kodande.
      \item Lärande är socialt, arbeta i grupper (max tre/grupp).
      \item Labbarna 1--2, 4--5 kamraträttas (peer review).
      \item Labbarna 3, 6 presenteras för assistent (examinerande).
    \end{itemize}
  \end{block}

  \pause

  \begin{remark}
    \begin{itemize}
      \item Skapa studiegrupper!
    \end{itemize}
  \end{remark}
\end{frame}

\begin{frame}
  \begin{remark}
    \begin{itemize}
      \item Alla tillfällen ges både online och på campus!
    \end{itemize}
  \end{remark}
\end{frame}

\begin{frame}
  \begin{block}{Fördjupande övningar}
    \begin{itemize}
      \item Interaktiva tillfällen för lärande.
      \item Fördjupande innehåll.
      \item Lämpligt för dem som har lätt för programmering.
      \item Ges endast online, men med salar på campus.
    \end{itemize}
  \end{block}
\end{frame}

%\begin{frame}
%  \begin{exercise}[Hur studerar ni?]
%    \begin{itemize}
%      \item Break-out rooms: tre nya personer i varje.
%      \item Hur har du studerat bäst hittills? Vad är din studieteknik?
%    \end{itemize}
%  \end{exercise}
%\end{frame}

\subsection{Hur kollar vi att ni kan?}

\begin{frame}
  \begin{block}<+>{Laborationer (LAB1, 1.5 hp)}
    \begin{itemize}
      \item Lämnas in i Canvas och presenteras muntligen (labb 3, 6).
      \item \emph{Man får misslyckas}, bara att fixa, lämna in och presentera 
        igen!
      \item Första halvan av kursen.
    \end{itemize}
  \end{block}

  \begin{block}<+>{Datorprov (LAB2, 1.5 hp)}
    \begin{itemize}
      \item Testar era kunskaper efter att vi täckt allt material.
      \item Mitten av kursen (tentaperioden).
      \item Heter Kontrollskrivning i schemat.
      \item Anmälan via Ladok, påminner er när det närmar sig.
    \end{itemize}
  \end{block}
\end{frame}

\begin{frame}
  \begin{block}{P-uppgift (LAB3, 3.0 hp)}
    \begin{itemize}
      \item Testar era färdigheter.
      \item Låter er utvecklas djupare.
      \item Andra halvan av kursen.
      \item Betyget baseras på denna.
    \end{itemize}
  \end{block}
\end{frame}

%\begin{frame}
%  \begin{block}{Matlablaborationen (MAT1, 1.5 hp)}
%    \begin{itemize}
%      \item Examinerar Matlabfärdigheter
%      \item Innan P-uppgiften drar igång.
%    \end{itemize}
%  \end{block}
%\end{frame}


\section{Disciplinära förseelser}

\mode<all>{\begin{frame}
  \includegraphics[width=\columnwidth]{fig/SverigesRikesLag.png}
\end{frame}

\begin{frame}
  \begin{block}{10 kap. Högskoleförordningen (1993:100)\nocite{SFS1993:100}}
    \textbf{Allmänna bestämmelser}

    \vspace{0.5em}
    1 §   Disciplinära åtgärder får vidtas mot studenter som
    \begin{enumerate}
      \item med otillåtna hjälpmedel eller på annat sätt försöker vilseleda vid 
        prov eller när en studieprestation annars ska bedömas,
      \item stör eller hindrar undervisning, prov eller annan verksamhet inom 
        ramen för utbildningen vid högskolan,
      \item stör verksamheten vid högskolans bibliotek eller annan särskild 
        inrättning inom högskolan, eller
      \item utsätter en annan student eller en arbetstagare vid högskolan för 
        sådana trakasserier eller sexuella trakasserier som avses i 1 kap. 4 § 
        diskrimineringslagen (2008:567).
    \end{enumerate}
  \end{block}
\end{frame}

\begin{frame}
  \begin{block}{10 kap. Högskoleförordningen (1993:100)\nocite{SFS1993:100}, 
    forts.}
    \textbf{Disciplinära åtgärder}

    \vspace{0.5em}
    2 § De disciplinära åtgärderna är varning och avstängning.

    \vspace{0.5em}
    Ett beslut om avstängning innebär att studenten inte får delta i 
    undervisning, prov eller annan verksamhet inom ramen för utbildningen vid 
    högskolan. Beslutet skall avse en eller flera perioder, dock sammanlagt 
    högst sex månader.

    \vspace{0.5em}
    Ett beslut om avstängning får också begränsas till att avse tillträde till 
    vissa lokaler inom högskolan.
  \end{block}
\end{frame}

\begin{frame}
  \begin{block}{10 kap. Högskoleförordningen (1993:100)\nocite{SFS1993:100}, 
    forts.}
    \textbf{Disciplinnämnden}

    \vspace{0.5em}
    3 § Ärenden om disciplinära åtgärder skall, om inte annat följer av 9 §, 
      handläggas av en disciplinnämnd. En sådan nämnd skall finnas vid varje 
      högskola.

    \vspace{0.5em}
    4 § Disciplinnämnden skall bestå av rektor som ordförande, en lagfaren 
      ledamot som skall vara eller ha varit ordinarie domare och en företrädare 
      för lärarna vid högskolan. Studenterna vid högskolan har rätt att vara 
      representerade i nämnden med två ledamöter. Förordning (1998:1003).
  \end{block}
\end{frame}

\begin{frame}
  \begin{block}{10 kap. Högskoleförordningen (1993:100)\nocite{SFS1993:100}, 
    forts.}
    13 §   När ett beslut om avstängning har fattats, skall underrättelser om 
       detta genast tillställas Centrala studiestödsnämnden och de organ inom 
       högskolan som berörs.
  \end{block}
\end{frame}

\begin{frame}
  \includegraphics[width=\columnwidth]{fig/uu-domstol.png}
\end{frame}

\begin{frame}
  \includegraphics[width=\columnwidth]{fig/hederskodex.png}
\end{frame}
}



\section{Kursens material}

\begin{frame}
  \begin{figure}
    \includegraphics[height=0.8\textheight]{canvas.png}
    \caption{Demo av Canvas}
  \end{figure}
\end{frame}

