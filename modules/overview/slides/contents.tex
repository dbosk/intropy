\title{%
  Intro till prgi (DD1315)
}
\author{Daniel Bosk}
\institute{%
  KTH EECS
}

\mode<article>{\maketitle}
\mode<presentation>{%
  \begin{frame}
    \maketitle
  \end{frame}
}

\mode*

\begin{abstract}
  According to variation theory (see~\cite{NCOL}\footfullcite{NCOL}), learning an 
educational objective requires the learner to distinguish all aspects of that 
educational objective. What these aspects are is hard to tell for someone who 
has already mastered the educational objective in question. However, 
misconceptions occur when the learner cannot yet distinguish one or more 
critical aspects. Thus misconceptions can help us identify what those 
(critical) aspects are. Then we can teach the learner to distinguish those 
critical aspects by varying examples through a series of patterns of variation.

In our work (in progress), we explore the existing literature on misconceptions 
in introductory programming courses and analyse it through the lens of 
variation theory to identify the necessary aspects of programming that a 
learner must learn to distinguish. We also outline patterns of variation to 
teach to distinguish these aspects.

\Textcite{NCOL} also connects the patterns of variation of variation theory to 
both deep learning and scientific discoveries. In both cases, the learners (the 
researcher is also a learner of the unknown) introduce variation for themselves 
through these patterns of variation. We hypothesize the connection between the 
patterns of variation and the skill of debugging (the programmer is learning 
about some unknown when debugging).

\end{abstract}


\section{Vilka är vi?}

\subsection{Lärare}

\begin{frame}
  \begin{block}{Kursansvarig lärare}
    \begin{itemize}
      \item Namn: Daniel Bosk
      \item Lärarutbildad i matematik och datateknik (CL, 2006--2011)
      \item Forskarutbildad i datalogi (2014--2020)
      \item Undervisat på universitetsnivå sedan 2011.
      \item Fokus på e-lärande och distansundervisning.
    \end{itemize}
  \end{block}
\end{frame}

\begin{frame}
  \begin{block}{Övningsledare}
    \begin{itemize}
      \item Aleks
      \item Emelie
      \item Celina
      \item Camilla
      \item John
      \item Karl
      \item Javid
      \item Linus
    \end{itemize}
  \end{block}
\end{frame}

\begin{frame}
  \begin{block}{Labbhandledare}
    \begin{itemize}
      \item Isak
      \item Wille
      \item Martin
      \item Tobias
    \end{itemize}
  \end{block}
\end{frame}

\subsection{Studenter}

\begin{frame}
  \begin{exercise}[Vad ha ni för intressen?]
    \begin{itemize}
      \item Gå till \url{menti.com} och ange koden 78 28 93 5.
    \end{itemize}
  \end{exercise}
\end{frame}

\begin{frame}
  \begin{exercise}[Vilka är ni?]
    \begin{itemize}
      \item Break-out rooms: tre personer i varje.
      \item Var kommer ni från?
      \item Varför kom ni hit?
      \item Var är ni på väg?
    \end{itemize}
  \end{exercise}
\end{frame}


\section{Om kursen}

\subsection{Vad ska ni få ut?}

\begin{frame}
  \begin{block}{Färdigheter}
    \begin{itemize}
      \item Algoritmiskt tänkande
      \item Problemlösningsförmåga
    \end{itemize}
  \end{block}

  \pause

  \begin{block}{Kunskaper}
    \begin{itemize}
      \item Terminologi
      \item Mental modell av datorsystem
    \end{itemize}
  \end{block}

  \pause

  \begin{block}{Verktyg}
    \begin{itemize}
      \item Python
      \item Matlab
    \end{itemize}
  \end{block}
\end{frame}

\subsection{Hur ska ni få ut detta?}

\begin{frame}
  \begin{block}{Moduler}
    \begin{enumerate}
      \item Förberedelse
      \item Föreläsning
      \item Laboration
      \item Föreläsning
      \item Övning
    \end{enumerate}
  \end{block}

  \pause

  \begin{block}{Projekt}
    \begin{itemize}
      \item Större uppgift.
      \item Andra halvan av terminen.
    \end{itemize}
  \end{block}
\end{frame}

\begin{frame}
  \begin{block}{Förberedelse}
    \begin{itemize}
      \item Egna studier.
      \item Interaktionen bra för lärande.
    \end{itemize}
  \end{block}

  \pause
  
  \begin{block}{Föreläsningar}
    \begin{itemize}
      \item Ger en översiktlig genomgång.
      \item Fokuserar på de viktigaste.
      \item Får information från interaktionen under era förberedelser.
    \end{itemize}
  \end{block}
\end{frame}

\begin{frame}
  \begin{block}{Laborationer}
    \begin{itemize}
      \item Driver ert lärande framåt.
      \item Lärande är socialt, arbeta i grupper om två.
    \end{itemize}
  \end{block}

  \pause

  \begin{remark}
    \begin{itemize}
      \item Skapa studiegrupper!
    \end{itemize}
  \end{remark}

  \pause

  \begin{block}{Övningar}
    \begin{itemize}
      \item Interaktiva tillfällen för lärande.
    \end{itemize}
  \end{block}
\end{frame}

\begin{frame}
  \begin{exercise}[Hur studerar ni?]
    \begin{itemize}
      \item Break-out rooms: tre nya personer i varje.
      \item Hur har du studerat bäst hittills? Vad är din studieteknik?
    \end{itemize}
  \end{exercise}
\end{frame}

\subsection{Hur kollar vi att ni kan?}

\begin{frame}
  \begin{block}{Laborationer}
    \begin{itemize}
      \item Lämnas in i Canvas.
      \item Måste även delta i övningen.
    \end{itemize}
  \end{block}

  \pause

  \begin{block}{Datorprov}
    \begin{itemize}
      \item Testar era kunskaper i slutet av perioden.
    \end{itemize}
  \end{block}

  \pause

  \begin{block}{P-uppgift}
    \begin{itemize}
      \item Testar era färdigheter.
      \item Låter er utvecklas djupare.
    \end{itemize}
  \end{block}
\end{frame}

%\begin{frame}
%  \begin{block}{Matlablaborationer}
%    \begin{itemize}
%      \item Examinerar Matlabfärdigheter
%    \end{itemize}
%  \end{block}
%\end{frame}


\section{Varför ska indek läsa programmering?}

\begin{frame}
  Presentation av indekare för indekare.
\end{frame}


\section{Kursens material}

\begin{frame}
  \begin{figure}
    \includegraphics[height=0.8\textheight]{canvas.png}
    \caption{Demo av Canvas}
  \end{figure}
\end{frame}

