\mode*

\section{En notis om pedagogik}

\begin{frame}
  \begin{question}
    \begin{itemize}
      \item Learn by trying or by being told?
    \end{itemize}
  \end{question}
\end{frame}

\begin{frame}
  \begin{example}[Learn by trying or by being told?
    {\cite[pp.~214--221]{NecessaryConditionsOfLearning}}]
    \begin{itemize}
      \item \Textcite{Szekely1950} studied the problem in 1950.
    \end{itemize}
    \begin{enumerate}
      \item<2,3> Presented a puzzling phenomena (Problem A) to a group.
      \item<2,3> Presented a text explaining the phenomena afterwards.
    \end{enumerate}
    \begin{enumerate}
      \item<3> Presented a text explaining the phenomena.
      \item<3> Presented the puzzling phenomena (Problem A) afterwards.
    \end{enumerate}
    \begin{itemize}
      \item<4> Called them back a week later with related Problem B.
      \item<4> The order for Problem A had significant impact on solving 
        Problem B.
    \end{itemize}
  \end{example}
\end{frame}

\begin{frame}
  \begin{remark}[Reproduced]
    \begin{itemize}
      \item Confirmed by later studies as well.
      \item \Textcite{BransfordSchwartz1999} studied effect on future learning.
      \item Students learn more easily from being told if they tried first.
    \end{itemize}
  \end{remark}

  \pause

  \begin{remark}
    \begin{itemize}
      \item \emph{Even if the time spent is the same!}
    \end{itemize}
  \end{remark}
\end{frame}


\section{Rekursion}

\begin{frame}
  \begin{exercise}[Skriv ett program]
    \begin{itemize}
      \item Vi vill skriva ut \enquote{hej} femtio gånger.
      \item Skriv ett program som gör det.
    \end{itemize}
  \end{exercise}
\end{frame}

\begin{frame}[fragile]
  \begin{solution}[En möjlig lösning]
    \begin{minted}{python}
      def skriv_ut(meddelande, antal_gånger):
        """Skriver ut meddelande antal_gånger antal gånger"""
        if antal_gånger <= 0:
          return

        print(meddelande)
        skriv_ut(meddelande, antal_gånger-1)
    \end{minted}
  \end{solution}
\end{frame}

\begin{frame}[fragile]
  \begin{definition}[Rekursion]
    \begin{minted}{python}
      def rekursiv_funktion(argument):
        """Rekurserar tills att vi når basfallet"""
        if basfall(argument):
          return

        rekursiv_funktion(nästa(argument))
    \end{minted}
  \end{definition}
\end{frame}

\begin{frame}[fragile]
  \begin{example}[Fibonacci]
    \begin{minted}{python}
      def fib(n):
        """Returnerar n:te talet i Fibonacci-serien"""
        if n <= 0:
          return 0
        elif n == 1:
          return 1

        return fib(n-1) + fib(n-2)
    \end{minted}
  \end{example}
\end{frame}

\begin{frame}[fragile]
  \begin{example}[Gåexemplet]
    \inputminted[firstline=32]{python}{examples/walk.py}
  \end{example}
\end{frame}


\section{Iterationer}

\begin{frame}[fragile]
  \begin{columns}[t]
    \begin{column}{0.45\columnwidth}
      \begin{minted}{python}
        for element in container:
          # do something to element
      \end{minted}
    \end{column}
    \begin{column}{0.55\columnwidth}
      \begin{minted}{python}
        while condition:
          # do something that might change
          # condition
      \end{minted}
    \end{column}
  \end{columns}
\end{frame}

\begin{frame}[fragile]
  \begin{columns}[t]
    \begin{column}{0.5\columnwidth}
  \begin{example}<1,4>[Alternativ: for]
    \begin{minted}{python}
      for gång in range(50):
        print("hej")
    \end{minted}
  \end{example}
    \end{column}

    \begin{column}{0.5\columnwidth}
  \begin{example}<2,4>[Alternativ: while]
    \begin{minted}{python}
      antal = 50
      while antal > 0:
        print("hej")
        antal -= 1
    \end{minted}
  \end{example}
    \end{column}
  \end{columns}

  \begin{example}<3,4>[Alternativ: rekursion]
    \begin{minted}{python}
      def skriv_ut(meddelande, antal_gånger):
        """Skriver ut meddelande antal_gånger antal gånger"""
        if antal_gånger <= 0:
          return
        print(meddelande)
        skriv_ut(meddelande, antal_gånger-1)

      skriv_ut("hej", 50)
    \end{minted}
  \end{example}
\end{frame}

\begin{frame}
  \begin{exercise}[Skriv ett program]
    \begin{enumerate}
      \item Låt användaren mata in ett positivt heltal.
      \item Skriv ut siffersumman av talet.
    \end{enumerate}
  \end{exercise}

  \begin{definition}<2>[Siffersumma]
    \begin{itemize}
      \item Siffersumman är summan siffrorna i ett tal.
      \item \(123\) har siffersumman \(1 + 2 + 3 = 6\).
    \end{itemize}
  \end{definition}

  \onslide<3>{
  \begin{question}
    \begin{itemize}
      \item Vilken typ av iteration är lämplig för de olika delarna?
    \end{itemize}
  \end{question}
}

  {\renewcommand\thefootnote{}
  \footnotetext{Idé: Olle Bälter}}
\end{frame}

\mode<all>{\endinput}

\section{Större exempel}

\subsection{Svårare gissningar, guess.py}

\begin{frame}[fragile]
  \lstinputlisting[linerange=5-10,firstnumber=5]{examples/guess.py}
\end{frame}

\begin{frame}[fragile]
  \lstinputlisting[linerange=11-25,firstnumber=11]{examples/guess.py}
\end{frame}

\begin{frame}[fragile]
  \lstinputlisting[linerange=11-13,firstnumber=11]{examples/guess.py}
  \lstinputlisting[firstline=24,firstnumber=24]{examples/guess.py}
\end{frame}


\subsection{Multiplikationskolumner, multcol.py}

\begin{frame}[fragile,allowframebreaks]
  \lstinputlisting[linerange=3-18,firstnumber=3]{examples/multcol.py}
\end{frame}

\begin{frame}[fragile,allowframebreaks]
  \lstinputlisting[firstline=19,firstnumber=19]{examples/multcol.py}
\end{frame}

\subsection{Multiplikationstabell 1--9, multtable.py}

\begin{frame}[fragile,allowframebreaks]
  \lstinputlisting[linerange=3-12,firstnumber=3]{examples/multtable.py}
\end{frame}

\begin{frame}[fragile,allowframebreaks]
  \lstinputlisting[firstline=13,firstnumber=13]{examples/multtable.py}
\end{frame}

\subsection{Multiplikationstabell godtycklig, multtable-expand.py}

\begin{frame}[fragile,allowframebreaks]
  \lstinputlisting[linerange=3-16,firstnumber=3]{examples/multtable-expand.py}
\end{frame}

\begin{frame}[fragile,allowframebreaks]
  \lstinputlisting[firstline=17,firstnumber=17]{examples/multtable-expand.py}
\end{frame}

