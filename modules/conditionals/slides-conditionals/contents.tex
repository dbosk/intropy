\title{%
  Villkor och styrstrukturer
}
\author{Daniel Bosk}
\institute{%
  KTH EECS
}

\mode<article>{\maketitle}
\mode<presentation>{%
  \begin{frame}
    \maketitle
  \end{frame}
}

\mode*

\begin{abstract}
  % What's the problem?
% Why is it a problem? Research gap left by other approaches?
% Why is it important? Why care?
% What's the approach? How to solve the problem?
% What's the findings? How was it evaluated, what are the results, limitations, 
% what remains to be done?

% XXX Summary
\emph{Summary:}
\dots

% XXX Motivation and intended learning outcomes
\emph{Intended learning outcomes:}
\dots

% XXX Prerequisites
\emph{Prerequisites:}
\dots

% XXX Reading material
\emph{Reading:}
\dots

\end{abstract}


\section{Datatyper och operationer}

\subsection{Numeriska typer}

\begin{frame}[fragile]
  \begin{block}{Operationer numeriska typer}
    \begin{itemize}
      \item \mintinline{python}|a + b| ger addition
      \item \mintinline{python}|a - b| ger subtraktion
      \item \mintinline{python}|a * b| ger multiplikation
      \item \mintinline{python}|a / b| ger division (heltal till flyttal)
      \item \mintinline{python}|a // b| ger heltalsdivision (heltal till heltal)
      \item \mintinline{python}|a % b| ger resten vid heltalsdivision
      \item \mintinline{python}|a ** b| ger \(a^b\)
    \end{itemize}
  \end{block}
\end{frame}

\subsection{Strängar}

\begin{frame}[fragile]
  \begin{block}{Operationer strängar}
    \begin{itemize}
      \item \mintinline{python}|a + b| ger sammanslagning
      \item \mintinline{python}|a * b| ger upprepning om \mintinline{python}|a| 
        är en sträng och \mintinline{python}|b| är ett heltal.
      \item \mintinline{python}|a[b]| ger en tecknet på position 
        \mintinline{python}|b|.
      \item \mintinline{python}|a[b:c]| ger en delsträng om 
        \mintinline{python}|b| och  \mintinline{python}|c| är heltal.
    \end{itemize}
  \end{block}

  \pause

  \begin{block}{Metoder strängar}
    \begin{itemize}
      \item \texttt{.capitalize()}
      \item \texttt{.center(width, fillchar)}
      \item \texttt{.find(substring, start, end)}
      \item \texttt{.strip()}
      \item \texttt{.upper()}, \texttt{.lower()}, \texttt{.casefold()}
    \end{itemize}
  \end{block}
\end{frame}

\subsection{Den booleska typen}

\begin{frame}
  \begin{center}
    \mintinline[fontsize=\Large]{python}|True or False|
  \end{center}
\end{frame}

\begin{frame}
  \begin{table}
    \begin{tabular}{rll}
      \mintinline{python}|or| & \mintinline{python}|True| & \mintinline{python}|False| \\
      \mintinline{python}|True| & \mintinline{python}|True| & \mintinline{python}|True| \\
      \mintinline{python}|False| & \mintinline{python}|True| & \mintinline{python}|False|
    \end{tabular}
    \caption{Sanningstabell för \mintinline{python}|or|.}
  \end{table}

  \pause

  \begin{table}
    \begin{tabular}{rll}
      \mintinline{python}|and| & \mintinline{python}|True| & \mintinline{python}|False| \\
      \mintinline{python}|True| & \mintinline{python}|True| & \mintinline{python}|False| \\
      \mintinline{python}|False| & \mintinline{python}|False| & \mintinline{python}|False|
    \end{tabular}
    \caption{Sanningstabell för \mintinline{python}|and|.}
  \end{table}
\end{frame}


\begin{frame}[fragile]
  \begin{block}{Jämförelseoperationer ger booleskt resultat}
    \begin{itemize}
      \item \mintinline{python}|a < b| sant om \(a\) är mindre än \(b\)
      \item \mintinline{python}|a <= b| sant som ovan eller om \(a\) lika med \(b\) 
        (\(\leq\))
      \item \mintinline{python}|a > b| sant om \(a\) större än \(b\)
      \item \mintinline{python}|a >= b| sant som ovan eller om \(a\) är lika med \(b\) 
        (\(\geq\))
      \item \mintinline{python}|a == b| sant om \(a\) och \(b\) är lika
    \end{itemize}
  \end{block}

  \pause

  \begin{remark}[Vanligt misstag]
    \begin{itemize}
      \item \mintinline{python}|a = b| är tilldelning
      \item \mintinline{python}|a == b| är jämförelse
    \end{itemize}
  \end{remark}
\end{frame}


\section{Villkor och styrstrukturer}

\subsection{Styrstrukturer}

\begin{frame}[fragile]
  \begin{minted}[fontsize=\Large]{python}
if villkor:
  print(True)
else:
  print(False)
  \end{minted}
\end{frame}

\begin{frame}[fragile]
  \begin{example}[\texttt{birthyear.py}]
    \inputminted[linenos]{python}{examples/birthyear.py}
  \end{example}
\end{frame}

\begin{frame}[fragile]
  \begin{minted}[fontsize=\large]{python}
if villkor1:
  print("villkor1 == True")
elif villkor2:
  print("villkor2 == True")
.
.
.
elif villkorN:
  print("villkorN == True")
else:
  print(False)
  \end{minted}
\end{frame}

\begin{frame}[fragile]
  \begin{example}
    \begin{minted}{python}
år = int(input("När föddes du?"))

if år < 2000:
    print("Du är i hyfsad form ändå.")
elif år < 1995:
    print("Du är gammal och grå!")
else:
    print("Du är ung och fräsch! Än så länge!")
    \end{minted}
  \end{example}

  \pause

  \begin{question}
    \begin{itemize}
      \item Funkar inte som tänkt, varför?
    \end{itemize}
  \end{question}
\end{frame}

\begin{frame}[fragile]
  \begin{example}[\texttt{birthyear.py}]
    \inputminted[linenos]{python}{examples/birthyear-elif.py}
  \end{example}
\end{frame}


\mode<all>{\endinput}

\section{Villkor och slingor}

\subsection{Slingor}

\begin{frame}[fragile]
  \begin{center}
    \begin{minted}[fontsize=\Large]{python}
while villkor:
    print(True)
print(False)
    \end{minted}
  \end{center}
\end{frame}

\begin{frame}[fragile]
  \begin{example}[Räkna ner]
    \begin{minted}{python}
n = 10
while n > 0:
  print(f"n = {n}")
  n -= 1
print("Done")
    \end{minted}
  \end{example}
\end{frame}

\begin{frame}[fragile]
  \begin{example}[Gissa talet]
    \begin{minted}{python}
guess = int(input("Guess what number I'm thinking of:"))
while guess != 2:
  guess = int(input("Wrong, guess again:"))
print("Finally, that's correct!")
    \end{minted}
  \end{example}
\end{frame}

\begin{frame}[fragile]
  \begin{example}[Mer gissande]
    \begin{minted}{python}
guess = int(input("Guess one of my favourite numbers:"))
while guess != 2 and guess != 3 and guess != 5:
  guess = int(input("Wrong, guess again:"))
print(f"That's correct, {guess} is one of my favourites.")
    \end{minted}
  \end{example}
\end{frame}

\begin{frame}[fragile]
  \begin{example}[Ännu mer gissande]
    \begin{minted}{python}
guess = int(input("Guess one of my favourite numbers:"))
while True:
  if guess == 2:
    break
  elif guess == 3:
    break
  elif guess == 5:
    break
  else:
    guess = int(input("Wrong, guess again:"))
print(f"That's correct, {guess} is one of my favourites.")
    \end{minted}
  \end{example}
\end{frame}

\begin{frame}[fragile]
  \begin{example}[Nog med gissande]
    \begin{minted}{python}
guess = int(input("Guess one of my favourite numbers:"))
while True:
  if guess == 2 or guess == 3 or guess == 5:
    break
  else:
    guess = int(input("Wrong, guess again:"))
print(f"That's correct, {guess} is one of my favourites.")
    \end{minted}
  \end{example}
\end{frame}

