\mode*

\section{Variabler}

\begin{frame}[fragile]
  \begin{block}{Variabler}
    \begin{itemize}
      \item Pekar på något i minnet.
    \end{itemize}
  \end{block}
  \begin{minted}{python}
    x = 5
    pi = 3.14
    name = "Adam"
    names = ["Adam", "Bertil", "Cesar"]
    phonebook = {"Adam": 07012345678,
                 "Bertil": 07212345678,
                 "Cesar": 07312345678,
                 "Östen": 07612345678}
  \end{minted}
\end{frame}

\begin{frame}[fragile]
  \begin{remark}
    \begin{itemize}
      \item Variabler kan peka på vad som helst.
    \end{itemize}
  \end{remark}

  \begin{minted}{python}
    def func(x):
      return 2*x

    y = func
    y(2) == func(2)
  \end{minted}
\end{frame}

\begin{frame}[fragile]
  \begin{block}{Datatyper}
    \begin{itemize}
      \item Allt har en datatyp!
      \item Datatyper: \mintinline{python}{int, float, str, list, dict} \dots
    \end{itemize}
  \end{block}

  \begin{minted}{python}
    x = 5 # int
    pi = 3.14 # float
    name = "Adam" # str
    names = ["Adam", "Bertil", "Cesar"] # list
    phonebook = {"Adam": 07012345678} # dict
    f = lambda x: x**2 # function
  \end{minted}
\end{frame}

\begin{frame}[fragile]
  \begin{remark}
    \begin{itemize}
      \item Olika typer har olika operationer.
    \end{itemize}
  \end{remark}

  \begin{minted}{python}
    x = 5
    y = 6
    z = x + y # int
    q = x / y # float
    firstname = "Adam"
    lastname = "Bertilsson"
    fullname = firstname + " " + lastname
  \end{minted}
\end{frame}

\begin{frame}[fragile]
  \begin{remark}
    \begin{itemize}
      \item Listor kan innehålla vad som helst.
    \end{itemize}
  \end{remark}

  \begin{minted}{python}
    strings = ["Adam", "Bertil", "Cesar"]
    ints = [1, 2, 3]
    funcs = [lambda x: x**2, lambda x: 2*x, lambda x: 2*x+1]
    lists = [strings, ints, funcs]
  \end{minted}
\end{frame}


\section{Strängar}

\begin{frame}[fragile]
  \begin{block}{Strängar}
    \begin{itemize}
      \item Innehåller text: \mintinline{python}{"Text"}
      \item Kan formateras med variabelinnehåll.
    \end{itemize}
  \end{block}

  \begin{minted}{python}
    name = "Adam"
    message1 = "Hej " + name + "!"
    message2 = f"Hej {name}!"
    message3 = "Hej, " + "hej, "*2 + f"på dig {name}!"
  \end{minted}
\end{frame}

\begin{frame}[fragile]
  \begin{remark}
    \begin{itemize}
      \item Med f-formatering kan vi göra mer än bara sätta in variabeln.
    \end{itemize}
  \end{remark}

  \begin{minted}{python}
    name = "Adam"
    age = 25
    message = f"Hej {name}! Du är {age:x} år i hexadecimalt format."

    for index in range(100):
      print(f"{index:3d}: data")
  \end{minted}
\end{frame}


\section{In- och utmatning}

\begin{frame}[fragile]
  \begin{block}{Kommunicera data}
    \begin{itemize}
      \item Via terminalen.
      \item Via filer.
    \end{itemize}
  \end{block}

  \begin{minted}{python}
    name = input("What's your name? ")
    print("Hej, {name}!")

    with open("fil.txt", "w") as file:
      file.write("Hej!\n")
    with open("fil.txt", "r") as file:
      print(file.read())
  \end{minted}
\end{frame}

\begin{frame}[fragile]
  \begin{remark}[\texttt{print}]
    \begin{itemize}
      \item Vi kan ändra \mintinline{python}{print}s beteende.
    \end{itemize}
  \end{remark}

  \begin{minted}{python}
    print("Hej!")
    print("Hej!", end="")

    with open("fil.txt", "w") as f:
      print("Hej!", file=f)
  \end{minted}
\end{frame}

\begin{frame}[fragile]
  \begin{remark}
    \begin{itemize}
      \item Kan läsa från filer på olika sätt.
    \end{itemize}
  \end{remark}

  \begin{minted}{python}
    with open("fil.txt", "r") as file:
      for line in file:
        print(line)

    with open("fil.txt", "r") as file:
      lines = file.readlines()
    for line in lines:
      print(line)

    with open("fil.txt", "r") as file:
      everything = file.read()
    lines = everything.split("\n")
  \end{minted}
\end{frame}


\section{Styrstrukturer}

\begin{frame}[fragile]
  \begin{block}{Styr programflödet}
    \begin{itemize}
      \item Gör olika saker beroende på vad som händer.
    \end{itemize}
  \end{block}

  \begin{minted}{python}
    name = input("What's your name? ")
    num = len(name)
    if num > 20:
      print("That's huge!")
    elif num > 10:
      print("That's big!")
    else:
      print("Keep it simple!")
  \end{minted}
\end{frame}


\section{Iterationer}

\begin{frame}[fragile]
  \begin{block}{Göra upprepade saker}
    \begin{itemize}
      \item Ofta behöver vi göra upprepningar.
      \item Vi kan göra det på två sätt.
    \end{itemize}
  \end{block}

  \begin{minted}{python}
    while condition:
      do_something()

    for item in items:
      do_something(item)
  \end{minted}
\end{frame}

\begin{frame}[fragile]
  \begin{minted}{python}
    num = int(input("Enter a number between 0--10: "))
    while num < 0 or num > 10:
      num = int(input("Enter a number between 0--10: "))

    for n in range(num):
      use_item(n)

    with open("fil.txt", "r") as file:
      for line in file:
        use_line(line)
  \end{minted}
\end{frame}


\section{Funktioner}

\begin{frame}[fragile]
  \begin{block}{Kodupprepning}
    \begin{itemize}
      \item Minimera kodupprepning.
      \item Koden finns bara på ett ställe.
    \end{itemize}
  \end{block}

  \begin{minted}{python}
    def input_interval(min_num, max_num):
      num = int(input("Enter a number between " +
                      f"{min_num}--{max_num}: "))
      while num < min_num or num > max_num:
        num = int(input("Enter a number between " +
                        f"{min_num}--{max_num}: "))

    num1 = input_interval(0, 10)
    num2 = input_interval(10, 20)
  \end{minted}
\end{frame}

\begin{frame}[fragile]
  \begin{block}{Moduler}
    \begin{itemize}
      \item Återanvända funktioner från andra pythonfiler.
    \end{itemize}
  \end{block}

  modulefile.py \hrulefill
  \begin{minted}{python}
    def square(x):
      return x**2
  \end{minted}

  \vspace{1em}
  mainfile.py \hrulefill
  \begin{minted}{python}
    import modulefile as mf

    def main():
      y = mf.square(2)

    if __name__ == "__main__":
      main()
  \end{minted}
\end{frame}

\begin{frame}[fragile]
  \begin{block}{Lambdafunktioner}
    \begin{itemize}
      \item Ibland vill vi ha små tillfälliga funktioner.
    \end{itemize}
  \end{block}

  \begin{minted}{python}
    xs = range(100)
    ys = map(lambda x: x**2,  xs)

    f = lambda x: 2*x+1
    print(f(2))
    print(f(3))
  \end{minted}
\end{frame}

generatorer
\begin{frame}[fragile]
  \begin{block}{Generatorer}
    \begin{itemize}
      \item Istället för att beräkna hela listan först, ta element för element.
    \end{itemize}
  \end{block}

  \inputminted[firstline=3,lastline=15]{python}{examples/generators.py}
\end{frame}


\section{Felhantering}

\begin{frame}[fragile]
  \begin{block}{Särfall}
    \begin{itemize}
      \item Programkod kastar särfall vid fel.
      \item Dessa måste hanteras annars kraschar programmet.
    \end{itemize}
  \end{block}

  \begin{minted}{python}
    try:
      int("a")
    except ValueError as err:
      print(f"We caught this: {err}")
  \end{minted}
\end{frame}

\begin{frame}[fragile]
  \begin{minted}{python}
    import input_type as it

    def get_age(limit):
      age = it.input_type(int, "What's your age? ")
      if age < limit:
        raise ValueError("You don't pass the age limit!")
      return age

    def main():
      age = get_age()
      print("Ah, nice, you're {age} years old!")
  \end{minted}
\end{frame}

\begin{frame}[fragile]
  xy.py \hrulefill
  \inputminted[linenos,firstline=3,lastline=10]{python}{examples/xy.py}
\end{frame}

\begin{frame}[fragile]
  xy.py \hrulefill
  \inputminted[linenos,firstline=12,lastline=16]{python}{examples/xy.py}
\end{frame}

\begin{frame}[fragile]
  xy.py \hrulefill
  \inputminted[linenos,firstline=18,lastline=28]{python}{examples/xy.py}
\end{frame}

\begin{frame}[fragile]
  xy.py \hrulefill
  \inputminted[linenos,firstline=30]{python}{examples/xy.py}
\end{frame}
