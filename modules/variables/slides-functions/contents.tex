\mode*

\section{Funktioner}

\subsection{Vad är bra med funktioner?}

\begin{frame}[fragile]
  \begin{exercise}
    \begin{itemize}
      \item Varför är det bra att använda funktioner?
    \end{itemize}
  \end{exercise}
\end{frame}

\begin{frame}[fragile]
  \begin{solution}[Bra med funktioner]
    \begin{itemize}
      \item Som \enquote{miniprogram} som går att återanvända.
      \item Gör att vi kan minimera kodupprepningar.
      \item Ger färre problem, underlättar underhåll och utbyggnad.
    \end{itemize}
  \end{solution}

  \pause

  \begin{example}[Att steka pannkakor{ {för fyra personer}}]
    \begin{enumerate}
    \item \alert<2>{Gör pannkakssmet} för fyra personer.
      \item För varje portion, medan det finns pannkakssmet kvar:
        \begin{enumerate}
          \item Häll \SI{1}{\deci\litre} smet i en het stekpanna.
          \item Vänta \SI{2}{\minute}.
          \item \alert<3>{Vänd} pannkakan.
          \item Vänta \SI{2}{\minute}.
          \item Servera pannkakan.
        \end{enumerate}
    \end{enumerate}
  \end{example}
\end{frame}

\subsection{Egna funktioner}

\begin{frame}[fragile]
    \begin{minted}[fontsize=\huge]{python}
def func(parameters):
  # use {parameters}
  return results
  \end{minted}
\end{frame}

\begin{frame}[fragile]
  \begin{example}[Addition]
  \begin{minted}[linenos]{python}
def add(x, y):
    """Returns x+y"""
    return x + y

print(f"2 + 3 = {2+3}")
print(f"2 + 3 = {add(2, 3)}")
    \end{minted}
  \end{example}
\end{frame}

\begin{frame}[fragile]
  \begin{example}[Dubblera]
    \begin{minted}[linenos]{python}
def double(x):
    """Returnerar dubbla x"""
    return 2*x

print(f"Dubbla 3 = {2*3}")
print(f"Dubbla 3 = {double(3)}")
    \end{minted}
  \end{example}
\end{frame}

\begin{frame}[fragile]
  \begin{example}[Celcius och Farenheit]
    \inputminted[linenos]{python}{examples/degees.py}
  \end{example}
\end{frame}

\begin{frame}[fragile]
  \begin{example}[Välkomsttext]
    \begin{minted}[linenos]{python}
def welcome(name, origin):
    """Returnerar en sträng med ett välkomstmeddelande baserat på
     - namnet i name,
     - avreseorten origin.
    """
    return f"Välkommen {name}, du som rest så långt som " \
           f"från {origin}!"

print(welcome("Daniel", "Kungsängen"))
print(welcome("Kungsängen", "Daniel")) # what?!
    \end{minted}
  \end{example}
\end{frame}

\begin{frame}[fragile]
  \begin{example}[Utan return, direkt till skärm]
    \begin{minted}[linenos,highlightlines={6,9-10}]{python}
def welcome_less_good(name, origin):
    """Skriver ut ett välkomstmeddelande baserat på
     - namnet i name,
     - avreseorten origin.
    """
    print(f"Välkommen {name}, du som rest så långt som "
          f"från {origin}!")

welcome("Daniel", "Kungsängen")
welcome("Kungsängen", "Daniel") # what?!
    \end{minted}
  \end{example}

  \begin{onlyenv}<2>
    \begin{exercise}
      \begin{itemize}
        \item Varför är denna version sämre?
      \end{itemize}
    \end{exercise}
  \end{onlyenv}
\end{frame}

\begin{frame}[fragile]
  \begin{example}[Välkomsttext med variabler]
    \begin{minted}[linenos,highlightlines={9-10,12-13}]{python}
def welcome(name, origin):
    """Returnerar en sträng med ett välkomstmeddelande baserat på
     - namnet i name,
     - avreseorten origin.
    """
    return f"Välkommen {name}, du som rest så långt som " \
           f"från {origin}!"

name = "Daniel"
origin = "Kungsängen"

print(welcome(name, origin))
print(welcome(origin, name)) # what?!
    \end{minted}
  \end{example}
\end{frame}

\begin{frame}[fragile]
  \begin{example}[Välkomsttext med variabler, andra namn]
    \begin{minted}[linenos,highlightlines={9-10,12-13}]{python}
def welcome(name, origin):
    """Returnerar en sträng med ett välkomstmeddelande baserat på
     - namnet i name,
     - avreseorten origin.
    """
    return f"Välkommen {name}, du som rest så långt som " \
           f"från {origin}!"

var1 = "Daniel"
var2 = "Kungsängen"

print(welcome(var1, var2))
print(welcome(var2, var1)) # what?!
    \end{minted}
  \end{example}
\end{frame}

\subsection{Dela upp i funktioner}

\begin{frame}[fragile]
  \begin{columns}[t]
    \begin{column}{0.3\columnwidth}
      \inputminted[linenos,firstline=11]{python}{../slides/examples/age.py}
    \end{column}
    \begin{column}{0.6\columnwidth}
      \begin{exercise}
        \begin{itemize}
          \item Dela upp i passande funktioner.
        \end{itemize}
      \end{exercise}
    \end{column}
  \end{columns}
\end{frame}

\begin{frame}[fragile]
  \inputminted[linenos,firstline=10,lastline=18]{python}{examples/age-funcs.py}
\end{frame}

\begin{frame}[fragile]
  \inputminted[linenos,firstline=10,lastline=10]{python}{examples/age-funcs.py}
  \inputminted[linenos,firstline=12,lastline=12]{python}{examples/age-funcs.py}
  \inputminted[linenos,firstline=16,lastline=16]{python}{examples/age-funcs.py}
  \inputminted[linenos,firstline=20,lastline=34]{python}{examples/age-funcs.py}
\end{frame}

\begin{frame}[fragile]
  \inputminted[linenos,firstline=12,lastline=12]{python}{examples/age-funcs.py}
  \inputminted[linenos,firstline=16,lastline=16]{python}{examples/age-funcs.py}
  \inputminted[linenos,firstline=20,lastline=20]{python}{examples/age-funcs.py}
  \vspace{0.5em}
  \inputminted[linenos,firstline=36]{python}{examples/age-funcs.py}
\end{frame}

